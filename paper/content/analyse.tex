\chapter{Analyse}\label{chapter:analysis}
%%%%%%%%%%%%%%%%%%%%%%%%%%%%%%%%%%%%%%%%%%%%%%%%%%%%%%%%%%%%

\imiscomment{Analyse der Problemstellung, der Zielgruppen sowie des Anwendungskontextes}

%%%%%%%%%%%%%%%%%%%%%%%%%%%%%%%%%%%%%%%%%%%%%%%%%%%%%%%%%%%%
\section{Problem- oder Aufgabenanalyse}\label{sec:problem_ana}
%%%%%%%%%%%%%%%%%%%%%%%%%%%%%%%%%%%%%%%%%%%%%%%%%%%%%%%%%%%%

\imiscomment{Problemanalyse oder Aufgabenanalyse (Aufgaben, die mit dem System von einem Benutzer gelösten werden können sollen)}

% The following is an example of an IMIS-styled table.
% You are free to use other styles, in fact, encouraged.
% You should look at other options, e.g. using booktable.sty
% Bringhurst says on the issue of rules etc. in tables:
%   There should be a minimum amount of furniture (rules,
%   boxes, dots and other guiderails for traveling through
%   typographic space) and a maximum amount of information.
% You cannot follow Bringhurst and IMIS at the same time,
% your choice.
\begin{table}[ht]
 \newcolumntype{Y}{>{\columncolor[gray]{.85}}>{\raggedright\arraybackslash}X}
  \begin{tabularx}{\columnwidth}{!{\vrule width 1pt}Y!{\vrule width 1pt}l|l|l|l|l|l|l!{\vrule width 1pt}}
    \noalign{\hrule height 1pt}
    \rowcolor[gray]{.85}
    \textbf{bis} & \textbf{15.5.} & \textbf{22.5.} & \textbf{5.6.} & \textbf{12.6.} & \textbf{19.6.} & \textbf{26.6.} & \textbf{3.7.} \\\noalign{\hrule height 1pt}
    \textbf{Content Theory}  & Lübeck & & Kiel     & Lübeck   & Final        &       & \\\hline
    \textbf{Content Videos}  & Lübeck & & Kiel     & Kiel     & Final        &       & \\\hline
    \textbf{Cover, Icon}     & Kiel   & & Lübeck   & Kiel     & Kiel         & Final & \\\hline
    \textbf{Videobearbeitung}& Lübeck & & SLH, LOS & BNR, IGS & CFW,CJB      & Final & \\\hline
    \textbf{Aufbereitung der Videos}
                             & Kiel   & & Kiel     & Lübeck   & Lübeck       & Final & \\\hline
    \textbf{Bilder/Videos der Interviews}
                             & Lübeck & & Lübeck   & Kiel     & Kiel/Lübeck  & Final & \\\hline
    \textbf{Audio-Sprechertexte}
                             & Lübeck & & Kiel     & Kiel     & Lübeck       & Final & \\
    \noalign{\hrule height 1pt}
  \end{tabularx}
  \caption{Beispieltabelle}
  \label{table:example}
\end{table}%

\begin{table}[ht]
  \begin{center}
    \begin{tabular}{lrr}
      \textbf{Beruf}  & \textbf{Anzahl} & \textbf{Anteil}\\ \hline
      Student/in         & 17 & 65.38\% \\
      Wissenschaftler/in &  3 & 11.54\% \\
      Professor/in       &  2 &  7.69\% \\
      Forscher/in        &  1 &  3.85\% \\
      Informatiker/in    &  1 &  3.85\% \\
      Führungskraft      &  1 &  3.85\% \\
      Verkaufskraft      &  1 &  3.85\% \\
    \end{tabular}
    \caption{Berufe der Umfrageteilnehmer}
    \label{table:berufe}
  \end{center}
\end{table}%

%%%%%%%%%%%%%%%%%%%%%%%%%%%%%%%%%%%%%%%%%%%%%%%%%%%%%%%%%%%%
\section{Benutzeranalyse}\label{sec:user_ana}
%%%%%%%%%%%%%%%%%%%%%%%%%%%%%%%%%%%%%%%%%%%%%%%%%%%%%%%%%%%%

\imiscomment{Zielgruppen}

%%%%%%%%%%%%%%%%%%%%%%%%%%%%%%%%%%%%%%%%%%%%%%%%%%%%%%%%%%%%
\section{Kontextanalyse}\label{sec:context_ana}
%%%%%%%%%%%%%%%%%%%%%%%%%%%%%%%%%%%%%%%%%%%%%%%%%%%%%%%%%%%%

\imiscomment{Beschreibung des räumlich-zeitlichen Umfeldes für den Einsatz des Systems}

%%%%%%%%%%%%%%%%%%%%%%%%%%%%%%%%%%%%%%%%%%%%%%%%%%%%%%%%%%%%
\section{Organisationsanalyse}\label{sec:orga_ana}
%%%%%%%%%%%%%%%%%%%%%%%%%%%%%%%%%%%%%%%%%%%%%%%%%%%%%%%%%%%%

\imiscomment{Beschreibung des organisatorischen Umfeldes für den Einsatz des Systems, also z.B. den betrieblichen Kontext bei einem System im Arbeitseinsatz}
\chapter{Dialogbeispiele}\label{chapter:dialogs}
%%%%%%%%%%%%%%%%%%%%%%%%%%%%%%%%%%%%%%%%%%%%%%%%%%%%%%%%%%%%
\section{Beschreibung einer Simulation}\label{sec:sim_description}
Zusammenfassendes Beispiel der Erstellung einer Simulation mit Hilfe von handgeschriebenen Java-Generatoren.\\
Dazu: ein UML Diagramm das die Abhängigkeiten der schon oben gezeigten Module untereinander zeigt und einleitet warum wir uns für diese feingliedrige Erstellung entschieden haben.\\
\section{Interaktion mit der SimpleGui}\label{sec:dialog_simplegui}


\imiscomment{Darstellung des Systems anhand von Beispieldialogen mit Abbildungen und Erläuterungen}

\imiscomment{Ausgiebige Verwendung von Fotos und Bildschirm-Harcopies}

\imiscomment{Bei Farbbildern sollte sichergestellt werden, dass diese auch in Schwarz-Weiss gut erkennbar sind, da selbst bei Produktion einer Arbeit in Farbe später eventuell Kopien angefertigt werden}

%%%%%%%%%%%%%%%%%%%%%%%%%%%%%%%%%%%%%%%%%%%%%%%%%%%%%%%%%%%%
%\section{Dialogbeispiel 1}\label{sec:dialog_1}
%%%%%%%%%%%%%%%%%%%%%%%%%%%%%%%%%%%%%%%%%%%%%%%%%%%%%%%%%%%%
%
% pdfLaTeX can use png, jpeg, etc:
% \begin{figure}[htb]
%   \begin{center}
%     \includegraphics[width=\textwidth]{pics/smartboardTablet}
%   \end{center}
%   \caption{Smartboard Software auf dem Tablett}
%   \label{fig:smartboardTablet}
% \end{figure}    

%%%%%%%%%%%%%%%%%%%%%%%%%%%%%%%%%%%%%%%%%%%%%%%%%%%%%%%%%%%%
%\section{Dialogbeispiel 2}\label{sec:dialog_2}
%%%%%%%%%%%%%%%%%%%%%%%%%%%%%%%%%%%%%%%%%%%%%%%%%%%%%%%%%%%%

%%%%%%%%%%%%%%%%%%%%%%%%%%%%%%%%%%%%%%%%%%%%%%%%%%%%%%%%%%%%
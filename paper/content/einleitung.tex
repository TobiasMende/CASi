\chapter{Einleitung}\label{chapter:introduction}
%%%%%%%%%%%%%%%%%%%%%%%%%%%%%%%%%%%%%%%%%%%%%%%%%%%%%%%%%%%%

\imiscomment{Einführung und Motivation des Themas}

Die Arbeit beschreibt die Entwicklung eines interaktiven Systems zur Verbesserung der Kommunikation und Kollaboration in Teams von Wissensarbeitern.

%%%%%%%%%%%%%%%%%%%%%%%%%%%%%%%%%%%%%%%%%%%%%%%%%%%%%%%%%%%%
\section{Ziele der Arbeit}\label{sec:goals}
%%%%%%%%%%%%%%%%%%%%%%%%%%%%%%%%%%%%%%%%%%%%%%%%%%%%%%%%%%%%

\imiscomment{Begründung für die Ziele und die Relevanz des Themas}

%%%%%%%%%%%%%%%%%%%%%%%%%%%%%%%%%%%%%%%%%%%%%%%%%%%%%%%%%%%%
\section{Stand der Technik}\label{sec:state_of_art}
%%%%%%%%%%%%%%%%%%%%%%%%%%%%%%%%%%%%%%%%%%%%%%%%%%%%%%%%%%%%

\imiscomment{Literaturrecherche und Darstellung anderer wichtiger Arbeiten zum Thema}

\imiscomment{Darstellung des ``State of the Art''}

\imiscomment{Zitiert werden soll in der Arbeit wie in folgendem Beispiel:}

Wie bereits mehrfach in der Literatur zu kontextualisierten, ambienten, erklärungsfähigen Systemen erläutert wurde \citep{Cassens-PhD-2008}, sind dabei vor allem Theorien und Modelle sozialer Interaktionen zwischen Mensch und Artefakt entscheidend. Auch \cite{Schmitt.ea-2010-mental_models_disappearing_systems} betrachten solche Systeme, hier vor allem zur Unterstützung der Kommunikation und Kollaboration in Teams. Ein wichtiger Aspekt ist auch die Intentionserkennung, die als neues Forschungsgebiet im Rahmen von Erklärungsfähigkeit und Ambienz begriffen werden kann. \citep{Kofod-Petersen.ea-2009-closed_doors} \gls{nemo}.

Eine hervorragend Möglichkeit diese Gedanken auszudrücken bietet das Textsatzsystem \LaTeX\ mit seinen vielen ergänzenden Paketen \citepweb{ctan-cat}. Andere Themen finden sich im Web \citepweb{imis}.

%%%%%%%%%%%%%%%%%%%%%%%%%%%%%%%%%%%%%%%%%%%%%%%%%%%%%%%%%%%%
\section{Vorgehensweise}\label{sec:approach}
%%%%%%%%%%%%%%%%%%%%%%%%%%%%%%%%%%%%%%%%%%%%%%%%%%%%%%%%%%%%

\imiscomment{Kurzer Überblick zur Vorgehensweise bei der Bearbeitung des Themas}

\imiscomment{Kurze Erläuterung, was in den einzelnen Kapitel dargestellt wird}


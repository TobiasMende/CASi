\chapter{Einleitung}\label{chapter:introduction}
%%%%%%%%%%%%%%%%%%%%%%%%%%%%%%%%%%%%%%%%%%%%%%%%%%%%%%%%%%%%

%\imiscomment{Einführung und Motivation des Themas}

Die Arbeit beschreibt die Planung und Entwicklung eines Simulators zur realitätsnahen Context Awareness Simulation. Die Motivation für diese Projekt ist vor allem die Entwicklung einer Testumgebung, die die Analyse des MACK Frameworks ohne hohe Hardwarekosten ermöglicht.

%%%%%%%%%%%%%%%%%%%%%%%%%%%%%%%%%%%%%%%%%%%%%%%%%%%%%%%%%%%%
\section{Ziele der Arbeit}\label{sec:goals}
%%%%%%%%%%%%%%%%%%%%%%%%%%%%%%%%%%%%%%%%%%%%%%%%%%%%%%%%%%%%

%\imiscomment{Begründung für die Ziele und die Relevanz des Themas}
Das Ziel unserer Arbeit war vor allem die Entwicklung eines modularen Simulators, der sowohl flexibel erweitert werden, als auch universell eingesetzt, werden kann.

%%%%%%%%%%%%%%%%%%%%%%%%%%%%%%%%%%%%%%%%%%%%%%%%%%%%%%%%%%%%
\section{Stand der Technik}\label{sec:state_of_art}
%%%%%%%%%%%%%%%%%%%%%%%%%%%%%%%%%%%%%%%%%%%%%%%%%%%%%%%%%%%%

\imiscomment{Literaturrecherche und Darstellung anderer wichtiger Arbeiten zum Thema}

%\imiscomment{Darstellung des ``State of the Art''}
Zum Zeitpunkt der Entwicklung gab es bereits einen Context Awareness Simulator, der umfangreiche und konfigurierbare Simulationen ermöglichte. Dieser Simulator besitzt jedoch keinerlei Schnittstellen für die Integration von Sensoren und Aktuatoren. Aus diesem Grund haben wir uns für die Entwicklung eines eigenständigen Simulators entschieden, da die Weiterentwicklung und Anpassung des bestehenden Simulators einen sehr hohen, schwer kalkulierbare Aufwand bedeutet hätte.

%\imiscomment{Zitiert werden soll in der Arbeit wie in folgendem Beispiel:}

%Wie bereits mehrfach in der Literatur zu kontextualisierten, ambienten, erklärungsfähigen Systemen erläutert wurde \citep{Cassens-PhD-2008}, sind dabei vor allem Theorien und Modelle sozialer Interaktionen zwischen Mensch und Artefakt entscheidend. Auch \cite{Schmitt.ea-2010-mental_models_disappearing_systems} betrachten solche Systeme, hier vor allem zur Unterstützung der Kommunikation und Kollaboration in Teams. Ein wichtiger Aspekt ist auch die Intentionserkennung, die als neues Forschungsgebiet im Rahmen von Erklärungsfähigkeit und Ambienz begriffen werden kann. \citep{Kofod-Petersen.ea-2009-closed_doors} \gls{nemo}.

%Eine hervorragend Möglichkeit diese Gedanken auszudrücken bietet das Textsatzsystem \LaTeX\ mit seinen vielen ergänzenden Paketen \citepweb{ctan-cat}. Andere Themen finden sich im Web \citepweb{imis}.

%%%%%%%%%%%%%%%%%%%%%%%%%%%%%%%%%%%%%%%%%%%%%%%%%%%%%%%%%%%%
\section{Vorgehensweise}\label{sec:approach}
%%%%%%%%%%%%%%%%%%%%%%%%%%%%%%%%%%%%%%%%%%%%%%%%%%%%%%%%%%%%
Bei der Planung des Projektes haben wir zunächst eine Anforderungsanalyse durchgeführt. Hierfür haben wir Gespräche mit Felix Schmitt und Jörg Cassens, den Hauptinteressenten, geführt, um ein tiefergehendes Verständnis vom MACK Framework, der MATe Implementierung und den Anforderungen an einen Simulator zu erhalten.

Im Anschluss haben wir noch Gespräche mit Olof-Joachim Frahm und Daniel Wilken geführt um weitere technische Einblicke in das Framework und laufende Arbeiten zu erhalten.

Nach den Gesprächen haben wir bestehende Simulatoren getestet und recherchiert, ob bereits Arbeiten zum Thema durchgeführt wurden. Die Recherchen führten zu dem Entschluss, ein Simulationsframework zu entwickeln, dass flexibler als bestehende Simulatoren ist, da diese unter Berücksichtigung der Anforderungen ungeeignet waren.

Im Anschluss an die Recherchen haben wir die Architektur und eine Beschreibungssprachen auf XML Basis geplant und uns unabhängig von der Architektur textuell drei Szenerien (Büro, Krankenhaus, Museum) überlegt, mit denen wir die Vollständigkeit und Flexibilität der Architektur verifizieren konnten.

Die eigentliche Entwicklung ist feature-driven und user-centered. Aus diesem Grund haben wir zunächst den Simulationskern und dann die einzelnen Komponenten entwickelt und unserem Betreuer zu den wöchentlichen Treffen eine aktuelle Version als jar-Datei bereitgestellt.
%\imiscomment{Kurzer Überblick zur Vorgehensweise bei der Bearbeitung des Themas}



In Kapitel \ref{chapter:analysis} beschreiben wir die Analysen, die der Planung und Entwicklung des Simulators vorausgegangen sind.

Die Projektplanung sowie die Konzeption der Architektur werden im Kapitel \ref{chapter:concept} beschrieben.

Über die Entwicklung des Simulationsframeworks schreiben wir in Kapitel \ref{chapter:realization}.

Eine detaillierte Beschreibung der Simulation, mit der die MATe Anwendung getestet werden kann, findet sich im Kapitel \ref{chapter:dialogs}.

Abschließend beschreiben wir im Kapitel \ref{chapter:eval} mit welchen Methoden wir die Qualität der Software und das Erreichen unserer Ziele sichergestellt haben und bringen in Kapitel \ref{chapter:conclusions} eine Zusammenfassung unserer Arbeit sowie Ausblicke auf offene Punkte und Erweiterungsmöglichkeiten des Simulators, die als Thema für weitere Projekte dienen könnten.

%\imiscomment{Kurze Erläuterung, was in den einzelnen Kapitel dargestellt wird}


\chapter{Einleitung}\label{chapter:introduction}
%%%%%%%%%%%%%%%%%%%%%%%%%%%%%%%%%%%%%%%%%%%%%%%%%%%%%%%%%%%%

%\imiscomment{Einführung und Motivation des Themas}

Die Arbeit beschreibt die Planung und Entwicklung eines Simulators zur realitätsnahen Context Awareness Simulation. Die Motivation für dieses Projekt ist vor allem die Entwicklung einer Testumgebung, die die Analyse des MACK Frameworks ohne hohe Hardwarekosten ermöglicht.

MACK ist ein Framework, welches am IMIS entwickelt wurde und das zur Konstruktion von Systemen, die Context Awareness bieten sollen, benutzt werden kann. Es besteht aus verschiedenen, modularen Komponenten, die in unterschiedlichen Kombination implementiert werden können. Beispiele für solche Systeme sind unter anderem folgende Szenarien:
\begin{itemize}
\item Ein Bürogebäude, in dem die Unterbrechbarkeit der Kollegen erkannt und sichtbar gemacht wird
\item Ein interaktives Museum, in dem sich die (elektronische) Führung an den Besucher anpasst
\item Ein ambientes System, in dem sich der Wohnraum an die Bewohner anpasst
\item Ein Krankenhaus, in dem das Personal über die Verfügbarkeit von anderen Mitarbeitern oder bestimmten Behandlungsräumen informiert wird
\end{itemize}
MACK ist aus dem Projekt MATe, das ebenfalls am IMIS ins Leben gerufen wurde, entstanden. MATe ist ein System, das in Büros benutzt werden soll, um die Unterbrechbarkeit der Mitarbeiter zu erkennen und anzuzeigen. Es soll außerdem die Möglichkeit bieten, bestimmte Nachrichten zwischen den Kollegen zu verschicken. Das System sammelt Daten mit Hilfe von Sensoren im Büro und verschiedene Reasoner versuchen anhand dieser Information die Unterbrechbarkeit und weitere Attribute der Teilnehmer festzustellen. Diese werden mittels Aktuatoren in der Umgebung erkennbar gemacht.

%%%%%%%%%%%%%%%%%%%%%%%%%%%%%%%%%%%%%%%%%%%%%%%%%%%%%%%%%%%%
\section{Ziele der Arbeit}\label{sec:goals}
%%%%%%%%%%%%%%%%%%%%%%%%%%%%%%%%%%%%%%%%%%%%%%%%%%%%%%%%%%%%

%\imiscomment{Begründung für die Ziele und die Relevanz des Themas}
Das Ziel unserer Arbeit ist vor allem die Entwicklung eines modularen Simulators, der sowohl flexibel erweitert werden, als auch universell eingesetzt, werden kann. Insbesondere soll das Programm so modular sein, dass nicht nur Systeme, die auf MACK basieren, damit getestet werden können.

Da das Programm aber voraussichtlich zunächst nur zur Simulation von MATe Komponenten benutzt wird und uns nur der MATe-Server zum Testen zur Verfügung steht, haben wir als Beispielsimulation eine Bürosimulation entworfen und vorerst auch nur Sensoren und Aktuatoren der MATe Anwendung implementiert.

Ein weiteres Ziel ist der Entwurf einer Beschreibungssprache, mit der man sowohl eine Simulation also auch neue Sensoren und Aktuatoren definieren kann. Die Planung dieser XML-basierten Sprache haben wir zwecks besserer Übersicht über den Umfang des Projektes zu Begin vorgenommen. Da der Fokus der Entwicklungsphase allerdings auf der Universalität und Stabilität des Simulators lag, haben wir die Implementierung der Sprache und eines entsprechenden Interpreters zurück gestellt, weshalb die Beschreibung der Simulationen vorerst nur in Java möglich ist.

%%%%%%%%%%%%%%%%%%%%%%%%%%%%%%%%%%%%%%%%%%%%%%%%%%%%%%%%%%%%
\section{Stand der Technik}\label{sec:state_of_art}
%%%%%%%%%%%%%%%%%%%%%%%%%%%%%%%%%%%%%%%%%%%%%%%%%%%%%%%%%%%%

%\imiscomment{Literaturrecherche und Darstellung anderer wichtiger Arbeiten zum Thema}

%\imiscomment{Darstellung des ``State of the Art''}
Zum Zeitpunkt der Entwicklung gab es bereits einen Context Awareness Simulator mit dem Namen \emph{Siafu}, der umfangreiche und konfigurierbare Simulationen ermöglicht. Dieser Simulator besitzt jedoch keinerlei Schnittstellen für die Integration von Sensoren und Aktuatoren. Aus diesem Grund haben wir uns für die Entwicklung eines eigenständigen Simulators entschieden, da die Weiterentwicklung und Anpassung des bestehenden Simulators einen sehr hohen, schwer kalkulierbare Aufwand bedeutet hätte. Ein weiterer Faktor für die Entscheidung zur Neuentwicklung war auch, dass der bestehende Simulator bereits seit 2007 nicht weiter entwickelt wurde und somit eine grundlegende Überarbeitung notwendig gewesen wäre.

%\imiscomment{Zitiert werden soll in der Arbeit wie in folgendem Beispiel:}

%Eine hervorragend Möglichkeit diese Gedanken auszudrücken bietet das Textsatzsystem \LaTeX\ mit seinen vielen ergänzenden Paketen \citepweb{ctan-cat}. Andere Themen finden sich im Web \citepweb{imis}.

%%%%%%%%%%%%%%%%%%%%%%%%%%%%%%%%%%%%%%%%%%%%%%%%%%%%%%%%%%%%
\section{Vorgehensweise}\label{sec:approach}
%%%%%%%%%%%%%%%%%%%%%%%%%%%%%%%%%%%%%%%%%%%%%%%%%%%%%%%%%%%%
Bei der Planung des Projektes haben wir zunächst eine Anforderungsanalyse durchgeführt. Hierfür haben wir Gespräche mit Felix Schmitt und Jörg Cassens, den Hauptinteressenten, geführt, um ein tiefgehendes Verständnis vom MACK Framework, der MATe Implementierung und den Anforderungen an einen Simulator zu erhalten.

Im Anschluss haben wir Gespräche mit Olof-Joachim Frahm und Daniel Wilken geführt, um weitere technische Einblicke in das Framework und laufende Arbeiten zu erhalten. Während der Entwicklungsphase haben wir uns Rückmeldungen von diesen Personen eingeholt.

Nach den Gesprächen haben wir bestehende Simulatoren getestet und recherchiert, ob bereits Arbeiten zu diesem Thema durchgeführt wurden. Die Recherchen führten zu dem Entschluss, ein Simulationsframework zu entwickeln, dass flexibler als bestehende Simulatoren ist, da diese unter Berücksichtigung der Anforderungen ungeeignet waren.

Im Anschluss an die Recherchen haben wir die Architektur und eine eigene Beschreibungssprache auf XML Basis geplant und uns unabhängig von der Architektur textuell drei Szenarien (Büro, Krankenhaus, Museum) geplant, mit denen wir die Vollständigkeit und Flexibilität der Architektur verifizieren konnten. Diese Vorbereitungen waren die Grundlage, auf der wir die Programmstruktur aufgesetzt haben.

Die eigentliche Entwicklung ist feature-driven und user-centered. Aus diesem Grund haben wir zunächst den Simulationskern und dann die einzelnen Komponenten entwickelt und unserem Betreuer zu den wöchentlichen Treffen eine aktuelle Version als jar-Datei bereitgestellt. So konnten wir frühzeitig die Rückmeldungen der späteren Anwender zu unseren Ansätzen bekommen und auf Wünsche und Anregungen der Benutzers reagieren.
%\imiscomment{Kurzer Überblick zur Vorgehensweise bei der Bearbeitung des Themas}



In Kapitel \ref{chapter:analysis} beschreiben wir die Analysen, die der Planung und Entwicklung des Simulators vorausgegangen sind.

Die Projektplanung sowie die Konzeption der Architektur werden im Kapitel \ref{chapter:concept} beschrieben.

Über die Entwicklung des Simulationsframeworks schreiben wir in Kapitel \ref{chapter:realization}.

Eine detaillierte Beschreibung der Simulation, mit der die MATe Anwendung getestet werden kann, findet sich im Kapitel \ref{chapter:dialogs}.

Abschließend beschreiben wir im Kapitel \ref{chapter:eval} mit welchen Methoden wir die Qualität der Software und das Erreichen unserer Ziele sichergestellt haben und bringen in Kapitel \ref{chapter:conclusions} eine Zusammenfassung unserer Arbeit sowie Ausblicke auf offene Punkte und Erweiterungsmöglichkeiten des Simulators, die als Thema für weitere Projekte dienen können.

%\imiscomment{Kurze Erläuterung, was in den einzelnen Kapitel dargestellt wird}


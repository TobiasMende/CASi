\chapter{Einleitung}\label{chapter:introduction}
%%%%%%%%%%%%%%%%%%%%%%%%%%%%%%%%%%%%%%%%%%%%%%%%%%%%%%%%%%%%

%\imiscomment{Einführung und Motivation des Themas}

Die Arbeit beschreibt die Planung und Entwicklung eines Simulators zur realitätsnahen Context Awareness Simulation. Die Motivation für diese Projekt ist vor allem die Entwicklung einer Testumgebung, die die Analyse des MACK Frameworks ohne hohe Hardwarekosten ermöglicht.

%%%%%%%%%%%%%%%%%%%%%%%%%%%%%%%%%%%%%%%%%%%%%%%%%%%%%%%%%%%%
\section{Ziele der Arbeit}\label{sec:goals}
%%%%%%%%%%%%%%%%%%%%%%%%%%%%%%%%%%%%%%%%%%%%%%%%%%%%%%%%%%%%

%\imiscomment{Begründung für die Ziele und die Relevanz des Themas}
Das Ziel unserer Arbeit war vor allem die Entwicklung eines modularen Simulators, der sowohl flexibel erweitert werden, als auch universell eingesetzt, werden kann.

%%%%%%%%%%%%%%%%%%%%%%%%%%%%%%%%%%%%%%%%%%%%%%%%%%%%%%%%%%%%
\section{Stand der Technik}\label{sec:state_of_art}
%%%%%%%%%%%%%%%%%%%%%%%%%%%%%%%%%%%%%%%%%%%%%%%%%%%%%%%%%%%%

\imiscomment{Literaturrecherche und Darstellung anderer wichtiger Arbeiten zum Thema}

%\imiscomment{Darstellung des ``State of the Art''}
Zum Zeitpunkt der Entwicklung gab es bereits einen Context Awareness Simulator, der bereits umfangreiche und konfigurierbare Simulationen ermöglichte. Dieser Simulator besitzt jedoch keinerlei Schnittstellen für die Integration von Sensoren und Aktuatoren. Aus diesem Grund haben wir uns für die Entwicklung eines eigenständigen Simulators entschieden, da die Weiterentwicklung und Anpassung des bestehenden Simulators einen sehr hohen, schwer kalkulierbare Aufwand bedeutet hätte.

%\imiscomment{Zitiert werden soll in der Arbeit wie in folgendem Beispiel:}

%Wie bereits mehrfach in der Literatur zu kontextualisierten, ambienten, erklärungsfähigen Systemen erläutert wurde \citep{Cassens-PhD-2008}, sind dabei vor allem Theorien und Modelle sozialer Interaktionen zwischen Mensch und Artefakt entscheidend. Auch \cite{Schmitt.ea-2010-mental_models_disappearing_systems} betrachten solche Systeme, hier vor allem zur Unterstützung der Kommunikation und Kollaboration in Teams. Ein wichtiger Aspekt ist auch die Intentionserkennung, die als neues Forschungsgebiet im Rahmen von Erklärungsfähigkeit und Ambienz begriffen werden kann. \citep{Kofod-Petersen.ea-2009-closed_doors} \gls{nemo}.

%Eine hervorragend Möglichkeit diese Gedanken auszudrücken bietet das Textsatzsystem \LaTeX\ mit seinen vielen ergänzenden Paketen \citepweb{ctan-cat}. Andere Themen finden sich im Web \citepweb{imis}.

%%%%%%%%%%%%%%%%%%%%%%%%%%%%%%%%%%%%%%%%%%%%%%%%%%%%%%%%%%%%
\section{Vorgehensweise}\label{sec:approach}
%%%%%%%%%%%%%%%%%%%%%%%%%%%%%%%%%%%%%%%%%%%%%%%%%%%%%%%%%%%%

\imiscomment{Kurzer Überblick zur Vorgehensweise bei der Bearbeitung des Themas}

\imiscomment{Kurze Erläuterung, was in den einzelnen Kapitel dargestellt wird}


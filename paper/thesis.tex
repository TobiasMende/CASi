\documentclass[11pt,    % standard font size
  english,ngerman,      % german primary, english secondary
  paper=a4,             % standard paper size
  oneside,              % one-sided print
  tablecaptionbelow,    % captions below tables
%   liststotocnumbered,   % dummy only, tocloft.sty will change it
%   bibtotocnumbered,     % dummy only, tocloft.sty will change it
%   BCOR=5mm,             % dummy only, geometry.sty will change it
%   normalheadings,       % dummy only, titlesec.sty will change it
%   headinclude,          % dummy only, geometry.sty will change it
%   footinclude,          % dummy only, geometry.sty will change it
%   draft,                % speed things up before final version
  DIV=calc              % dummy only, geometry.sty will change it
  ]{scrbook}            % KOMA script book class

% The dummy settings would be sensible for use with KOMA without
% modifications for use at IMIS. They will, however, be overwritten
% by other packages later to give the document an IMIS look.

%%%%%%%%%%%%%%%%%%%%%%%%%%%%%%%%%%%%%%%%%%%%%%%%%%%%%%%%%%%%
% This is an example LaTeX document for use with theses at
% the Institute for Multimedia and Interactive System (IMIS)
% at the University of Lübeck.
%
% It is unofficial, but if used without changes, it mimics
% the style of the official MS Word template quite closely.
% Typographically dubious choices in this template are
% due to shortcomings of the Word template.
%
% Adapt it where you see fit - you don't have to follow all
% idiosyncrasies of the default style, but can use
% typographically sane variations.
%
% You should be able to use it as it is, parts where your
% have to change things or decide on options are marked:
% FIXME change something here
%
% This template should only need packages available in a
% standard TeXlive install. Inputencoding is UTF 8, please
% make sure your editor gets that right. We use pdfLaTeX
% and produce PDF directly, we do not go via DVI.
%
% Happy TeXing:)
%
% Please direct comments, feedback, patches to:
% cassens@imis.uni-luebeck.de
%
% ChangeLog
% 2010-09-23 project started (jc)
% 2010-09-29 initial release (jc)
% 2010-09-30 bugfix imis.bst (jc)
% 2010-10-08 hacks for listings added (jc)
% 2010-11-16 changed titlepage as in new word template (jc)
% 2011-02-02 experimental workaround listings in german (jc)
% 2011-06-17 changed way of referencing weblinks (jc)
% 2011-07-29 removed broken package ltablex and use plain
%            tabularx. Beware: tabularx does not support
%            tables over several pages (thanks Niels for bug
%            report & Bjoern for debugging) (jc)
%%%%%%%%%%%%%%%%%%%%%%%%%%%%%%%%%%%%%%%%%%%%%%%%%%%%%%%%%%%%

% this makes the references a section instead of a chapter
% unfortunately, this has to be done before other packages
% mess around with the definition
\makeatletter
\renewcommand*\bib@heading{%
  \section*{\bibname}%
  \@mkboth{\bibname}{\bibname}%
}
\makeatother

% First we load the needed packages
%%%%%%%%%% General      %%%%%%%%%%
% language settings, input- and output-encoding
% localization (laguages set in document directive):
\usepackage{babel}
% we use UTF8 input:
\usepackage[utf8]{inputenc}
% out put encoding:
\usepackage[T1]{fontenc}

%%%%%%%%%% Fonts        %%%%%%%%%%
% we use a mix of Times,
% Courier and Helvetica:
\usepackage{mathptmx,courier}
% non-scaled Helvetica has wrong size
% (I know Times and Helvetica do not really
% fit together, but there is no fitting sans
% font in the default LaTeX installs)
\usepackage[scaled]{helvet}

%%%%%%%%%% Look & Feel  %%%%%%%%%%
% nicer headings/footer:
\usepackage{scrpage2}
% needed to change appearance of titles:
\usepackage{titlesec}
% typeset URLs nicely:
\usepackage{url}
% extended table environment:
\usepackage{tabularx}
% colors in tables:
\usepackage{colortbl}
% now we define the margins:
\usepackage[includeheadfoot, % footer/heading to text area
  inner=3cm,%     % inner margin larger for bindig correction
  outer=2.5cm,%   % outer margin
  top=2.5cm,%     % top margin
  bottom=1.5cm]%  % bottom margin
  {geometry}
% increases the line spacing:
\usepackage[onehalfspacing]{setspace}
% running fgure/table numbers, but per chapter:
\usepackage{chngcntr}
% used to adapt listoffigures, listoftables:
\usepackage{tocloft}
% used for adapting list environment, e.g. for glossaries
\usepackage{enumitem}
% include source code listings in the text
\usepackage{listings}

% define new float environments
\usepackage{float}


%%%%%%%%%% Graphics     %%%%%%%%%%
% we need to be able to include graphics:
\usepackage{graphicx}
% we use colors in tables etc.:
\usepackage{xcolor}
% TikZ is cool for diagrams:
\usepackage{tikz}
\usetikzlibrary{arrows,positioning}
%%%%%%%%%% Bib          %%%%%%%%%%
% allows us to use author-year styles
\usepackage{natbib}
% allows to have a separate bibliography for weblinks (or more)
\usepackage{multibib}

%%%%%%%%%% Other        %%%%%%%%%%
% % is already loaded automatically:
% \usepackage{calc}
% % do we want an index?
% \usepackage{makeidx}
% include hyperlinks in pdf and much more:
% Links in PDF versions going to the
% examiner should be black, else you
% might prefer the colored version
% commented out below.
\usepackage[
  colorlinks=true,
  linkcolor=black,
  pagecolor=black,
  menucolor=black,
  citecolor=black,
  urlcolor=black,
%   linkcolor=red!40!black,
%   pagecolor=red!40!black,
%   menucolor=red!40!black,
%   citecolor=blue!50!black,
%   urlcolor=green!40!black,
]{hyperref}
% beware: the hyperref package should always be loaded as the
% last one since it redefines a lot of commands. There are
% a few exceptions to this rule, e.g. when using bibunits.sty
% another exception is this one for building glossaries:
\usepackage[toc,% % entry in table of contents
  acronym,%       % we want a list of acronyms
  nonumberlist]%  % we don't need backlinks
  {glossaries}

% Now all packages are loaded and we can start to modify
% the look and feel.

%%%%%%%%%%%%%%%%%%%%%%%%%%%%%%%%%%%%%%%%%%%%%%%%%%%%%%%%%%%%
% Set up header and footer
%%%%%%%%%%%%%%%%%%%%%%%%%%%%%%%%%%%%%%%%%%%%%%%%%%%%%%%%%%%%
% % this is the default for IMIS:
% % left headers, page numbers to the right
% \pagestyle{scrheadings}
% \clearscrheadfoot
% \renewcommand{\headfont}{\normalfont\rmfamily\itshape}
% \ihead{\headmark}
% \ofoot[\pagemark]{\pagemark}

% You can try this alternative:
\pagestyle{scrheadings}
\clearscrheadfoot
\renewcommand{\headfont}{\normalfont\rmfamily\itshape}
\ohead{\headmark}
\ofoot[\pagemark]{\pagemark}

%%%%%%%%%%%%%%%%%%%%%%%%%%%%%%%%%%%%%%%%%%%%%%%%%%%%%%%%%%%%
%   General changes to formatting
%%%%%%%%%%%%%%%%%%%%%%%%%%%%%%%%%%%%%%%%%%%%%%%%%%%%%%%%%%%%
% this defines how the listing environments (for source code)
% should look like
% for producing the captions for the listings and the list of
% listings in the end, we have two options:
% a) rely on the built-in macros of listings.sty, using
%    \caption and \lstlistoflistings
%    Pro: Easy to use
%    Con: Both the captions and the list of listings
%         will look different to the ones used for
%         tables and figures. In order to make them look
%         identical, we would need to re-define a lot of
%         internal commands from listings.sty, that could
%         easily break with updates to listings.sty
% b) build our own \illcaption command and encapsulate the
%    lstlistings in a new float environment, illfloat
%    Pro: Captions and list of figures look coherent with
%         tables and figures
%    Con: Every lstlisting environment needs to be inside
%         an illfloat float environment to make sure the
%         listing and the caption are not separated. We can
%         use the [H] placement modifier from float.sty to
%         make sure the listing shows up exactly where
%         we define it, but we loose the option to have
%         pagebreaks in long listings.
% Chose your poison. Since the standard way of doing is
% well documented in the listings.sty documentation, I will
% show how to use option b) in this example file.
% First some basic styling
% FIXME you will probably not use Pascal
\lstset{frame=single,language=Java,captionpos=b,nolol=false}
% TODO differences in defined languages for listings.sty
\AtBeginDocument{%
  \makeatletter
    \addto\captionsngerman{%
      \def\lstlistingname{Quelltext}%
      \def\lstlistlistingname{Quelltexte}%
    }%
  \makeatother
}
% defining new commands for in-line emphazising of
% code classes and methods.
\definecolor{codeinlinecolor}{RGB}{0,0,0} 
\definecolor{codeclasscolor}{RGB}{0,0,100}
\definecolor{codemethodcolor}{RGB}{0,100,0}
\newcommand{\codeinline}[1]{\texttt{{\color{codeinlinecolor} #1}}}
\newcommand{\codeinterface}[1]{\texttt{{\color{codeclasscolor} #1}}}
\newcommand{\codeclass}[1]{\texttt{{\color{codeclasscolor} #1}}}
\newcommand{\codemethod}[1]{\texttt{{\color{codemethodcolor} #1}}}

% % for other than KOMA classes, the following may have
% % to be used instead:
% \renewcommand*{\lstlistlistingname}{Quelltexte}
% \renewcommand*{\lstlistingname}{Quelltexte}

% we use tocloft.sty to add Abbildung or Tabelle
% in front of entries in their respective lists of ...
% diversion from Word template: Tabelle and Abbildung
% are not set in bold in the list of ...
% we also take the -verzeichnis out of the names
% TODO babel for listoffigures/tables
% first the listoffigures:
\newlength\imisfiglength
\settowidth\imisfiglength{Abbildung\ }
\addtolength\cftfignumwidth{\imisfiglength}
\addtolength\cftfignumwidth{-1em}
\setlength{\cftfigindent}{0em}
\renewcommand{\cftfigpresnum}{Abbildung\ }
\renewcommand{\cftfigaftersnum}{:}
\addto\captionsngerman{%
  \renewcommand{\listfigurename}%
    {Abbildungen}%
}
% now the listoftables:
\newlength\imistablength
\settowidth\imistablength{Tabelle\ }
\addtolength\cfttabnumwidth{\imistablength}
\addtolength\cfttabnumwidth{-1em}
\setlength{\cfttabindent}{0em}
\renewcommand{\cfttabpresnum}{Tabelle\ }
\renewcommand{\cfttabaftersnum}{:}
\addto\captionsngerman{%
  \renewcommand{\listtablename}%
    {Tabellen}%
}
% now the listoflistings with tocloft (option b))
% described above - needed when we want the captions
% and list of listings to look similar to  those
% for figures and tables
% First we define a new list:
\newcommand{\listimislistingsname}{Quelltexte}
\newlistof{imislistings}{ill}{\listimislistingsname}
% then we format it similar to figures/tables
\newlength\imisilllength
\settowidth\imisilllength{Quelltext\ }
\addtolength\cftimislistingsnumwidth{\imisilllength}
\setlength{\cftimislistingsindent}{0em}
\renewcommand{\cftimislistingspresnum}{Quelltext\ }
\renewcommand{\cftimislistingsaftersnum}{:}

% We define a new float environment to keep the lstlistings
% and illcaption together - remember you can keep the special
% placement option [H] to keep the listings in place in your
% text, like you do when not using the [float] option of the
% lstlisting-environment:
\floatstyle{komabelow}
\newfloat{illfloat}{htb}{ilf}
\floatname{illfloat}{Quelltext}

% now we define a new \illcaption-macro to be used
% together with the lstlistings:
\makeatletter
\newcommand{\illcaption}[1]{%
\@ifnextchar[{\illcaption@i{#1}}{\illcaption@i{#1}[{#1}]}%
}
\def\illcaption@i#1[#2]{%

  \refstepcounter{imislistings}
%   \newcolumntype{Y}{>{\raggedright\arraybackslash}X}
  \begin{tabularx}{\columnwidth}{rX}
    \textbf{Quelltext \theimislistings:} & #1 \\
  \end{tabularx}

  \addcontentsline{ill}{imislistings}{\protect\numberline{\theimislistings}#2}\par
}
\makeatother

% the rest of the formatting changes is easy in comparison:

% we want no identation of the first line of a paragraph,
% but a skip between paragraphs:
\parindent 0pt
\parskip 6pt

% we number figures and tables throughout the text
% using the chngcntr.sty package:
\counterwithout{figure}{chapter}
\counterwithout{table}{chapter}

% labels in captions for tables and figures screaming
% in bold:
\renewcommand*{\figureformat}{\textbf{\figurename~\thefigure\autodot}}
\renewcommand*{\tableformat}{\textbf{\tablename~\thetable\autodot}}

% this is the KOMA-way of changing titles. We only
% use it to format the toc, rest is done below:
\setkomafont{sectioning}{\rmfamily\bfseries}

% Now we define how the different section headings look like.
% These settings use the titlesec.sty package:
\titleformat{\chapter}          % which level
  {\relax}                      % shape of the heading
  {\huge\textrm{\thechapter}}   % format of mark
  {1em}                         % separate mark and title
  {\huge\rmfamily}              % format of title

\titleformat{\section}
  {\relax}
  {\Large\textrm{\thesection}}
  {1em}
  {\Large\rmfamily}

\titleformat{\subsection}
  {\relax}
  {\large\textrm{\thesubsection}}
  {1em}
  {\large\rmfamily}

\titleformat{\subsubsection}
  {\relax}
  {\normalsize\textit{\thesubsubsection}}
  {1em}
  {\normalsize\itshape}

\titleformat{\paragraph}
  [runin]
  {\normalsize\bfseries}
  {\theparagraph}
  {0pt}
  {}

\titleformat{\subparagraph}
  [runin]
  {\normalsize\itshape}
  {\theparagraph}
  {0pt}
  {}

%%%%%%%%%%%%%%%%%%%%%%%%%%%%%%%%%%%%%%%%%%%%%%%%%%%%%%%%%%%%
% Set up citation styles
%%%%%%%%%%%%%%%%%%%%%%%%%%%%%%%%%%%%%%%%%%%%%%%%%%%%%%%%%%%%

% we use multibib.sty to generate a weblinks bibliography
% define more as you see fit (Software, RFCs. Pictures)
\newcites{web}{Weblinks}
% % If you use several bibliographies, you need to run all
% % of them through bibTeX. This can be achieved with the
% % following script (for Unix, Mac, and Linux):
% #!/bin/bash
% for file in *.aux ; do
%   bibtex `basename $file .aux`
% done

% Indentation in the bibliography is supposed to be
% quite large at IMIS
\setlength{\bibhang}{1cm}

% do we want to include a glossary and an index?

\makeglossary
% \makeindex

%%%%%%%%%%%%%%%%%%%%%%%%%%%%%%%%%%%%%%%%%%%%%%%%%%%%%%%%%%%%
% glossaries and indices
%%%%%%%%%%%%%%%%%%%%%%%%%%%%%%%%%%%%%%%%%%%%%%%%%%%%%%%%%%%%

% these files are used for defining abbreviations and gloses
% FIXME you have to run makeglossaries for creating glossaries
% makeglossaries is a perl script included with glossaries.sty
% read the documentation for further info
\newglossaryentry{ontologie}{
  name=Ontologie,
  description={Form der Wissensrepräsentation, legt Objekte und Relationen fest. Sie ist besonders für den Autausch geeignet und legt Basis einer gemeinsamen Konzeptualisierung}}
\newglossaryentry{aktor}{
  name=Aktor,
  description={Natürlicher oder künstlicher Teilnehmer an Prozessen}}
\newglossaryentry{eac}{
  name=Explanation-Aware Computing,
  description={Forschungsgebiet welches sich mit informatischen Systems beschäftigt welche in der Lage sind Erklärungen ihres Verhaltens zu geben oder solche Erklärungen von Seiten des Benutzers zu verarbeiten}}
\newacronym{casi}{CASi}{Context Awareness Simulator}
\newacronym{mate}{MATe}{Mate for Awareness in Teams}
\newacronym{casix}{CASiX}{CASi XML Simulationsbeschreibungssprache}

% we define two styles, for abbreviations and glossaries
% we make use of enumitem.sty, you might have to change
% the margins to fit your needs

\newglossarystyle{imis}{%
  \glossarystyle{list}%
  \renewenvironment{theglossary}%
    {\begin{description}[font=\rmfamily\bfseries,leftmargin=4cm,style=multiline]}%
    {\end{description}}%
  }

\newglossarystyle{imisabk}{%
  \glossarystyle{list}%
  \renewenvironment{theglossary}%
    {\begin{description}[font=\normalfont,leftmargin=2cm,style=multiline]}%
    {\end{description}}%
  }

%%%%%%%%%%%%%%%%%%%%%%%%%%%%%%%%%%%%%%%%%%%%%%%%%%%%%%%%%%%%
% float control
%%%%%%%%%%%%%%%%%%%%%%%%%%%%%%%%%%%%%%%%%%%%%%%%%%%%%%%%%%%%
% more sensible defaults for how much space floats can take
% up on pages with text:
% http://www.tug.org/texmf-dist/doc/generic/FAQ-en/html/FAQ-floats.html
\setcounter{topnumber}{2}
\setcounter{bottomnumber}{9}
\setcounter{totalnumber}{20}
\setcounter{dbltopnumber}{9}
\renewcommand{\topfraction}{0.85}
\renewcommand{\bottomfraction}{0.7}
\renewcommand{\textfraction}{0.15}
\renewcommand{\floatpagefraction}{0.7}
\renewcommand{\dbltopfraction}{.7}
\renewcommand{\dblfloatpagefraction}{.7}

%%%%%%%%%%%%%%%%%%%%%%%%%%%%%%%%%%%%%%%%%%%%%%%%%%%%%%%%%%%%
% some commands necessary for the template
%%%%%%%%%%%%%%%%%%%%%%%%%%%%%%%%%%%%%%%%%%%%%%%%%%%%%%%%%%%%
% the imiscomment command outputs some hints about what to
% write about in the different parts of the thesis
% you should either delete them all or redefine this
% command to an empty one
\definecolor{imiscommentcolor}{RGB}{142,0,0}
\newcommand{\imiscomment}[1]{\textit{\color{imiscommentcolor}#1}}

%%%%%%%%%%%%%%%%%%%%%%%%%%%%%%%%%%%%%%%%%%%%%%%%%%%%%%%%%%%%
% let the fun begin
%%%%%%%%%%%%%%%%%%%%%%%%%%%%%%%%%%%%%%%%%%%%%%%%%%%%%%%%%%%%
\begin{document}
\lstset{language=Java}
% stuff before real content, no chaptermarks, roman page numbers
\frontmatter

% % the cite commands will generate entries in the index, if used
% \citeindextrue

%%%%%%%%%%%%%%%%%%%%%%%%%%%%%%%%%%%%%%%%%%%%%%%%%%%%%%%%%%%%
% We don't use a custom \maketitle for now
% so we have to make our own titlepage
%%%%%%%%%%%%%%%%%%%%%%%%%%%%%%%%%%%%%%%%%%%%%%%%%%%%%%%%%%%%
\thispagestyle{empty}
\vspace*{-2.2cm}
\begin{center}
  \includegraphics[width=2.9cm]{pics/uni-siegel.pdf}\\

  \vspace{.5cm}
  {
  \fontsize{16pt}{10pt}\selectfont
  Universität zu Lübeck\\
  }

  \vspace{.65cm}
  {
    \fontsize{14pt}{19pt}\selectfont
    Institut für Multimediale und Interaktive Systeme\\
    Direktor: Prof. Dr. rer. nat. Michael Herczeg\\
  }
  \vspace{2.5cm}
  {
    \fontsize{22pt}{22pt}\selectfont
    \bfseries
    CASi - Context Awareness Simulator \\}
  \vspace{1.5cm}
  {
    \fontsize{16pt}{16pt}\selectfont
    Projektarbeit\\
  }
  \vspace{1.5cm}
  {
    \fontsize{13pt}{13pt}\selectfont
    vorgelegt von\\
    \vspace{10pt}
  Moritz Bürger, Marvin Frick und Tobias Mende\\
  }
  \vspace{2.5cm}
  {
    \fontsize{13pt}{13pt}\selectfont
    Prüfer\\
    \vspace{10pt}
    Prof. Dr. rer. nat. Michael Herczeg\\
  }
  \vspace{1.5cm}
  {
    \fontsize{13pt}{13pt}\selectfont
    wissenschaftliche Begleitung\\
    \vspace{10pt}
    Felix Schmitt, M.A.
  }
\end{center}
%%%%%%%%%%%%%%%%%%%%%%%%%%%%%%%%%%%%%%%%%%%%%%%%%%%%%%%%%%%%

%%%%%%%%%%%%%%%%%%%%%%%%%%%%%%%%%%%%%%%%%%%%%%%%%%%%%%%%%%%%
\chapter*{Kurzfassung}
\thispagestyle{empty}
%%%%%%%%%%%%%%%%%%%%%%%%%%%%%%%%%%%%%%%%%%%%%%%%%%%%%%%%%%%%

In unserer Projektarbeit im Rahmen des Praktikums im Bereich Interaktions- und Mediengestaltung beschäftigen wir uns mit der Konzeption und Entwicklung eines Context Awareness Simulators.

Ziel des Projektes ist es, eine Simulation zu entwickeln, mit der das MACK Framework und die dazugehörigen Reasoner ohne spezielle Hardware getestet werden können. Hierfür entwickeln wir zunächst ein abstraktes Simulationsframework, mit dem Simulationen für verschiedene Umgebungen entwickelt und simuliert werden können.

Auf dieser Grundlage implementieren wir eine Simulation zum Test der MATe Anwendung.

\vfill

%%%%%%%%%%%%%%%%%%%%%%%%%%%%%%%%%%%%%%%%%%%%%%%%%%%%%%%%%%%%
\section*{Schlüsselwörter} Simulator, Context Awareness, Software Entwicklung
%%%%%%%%%%%%%%%%%%%%%%%%%%%%%%%%%%%%%%%%%%%%%%%%%%%%%%%%%%%%


\selectlanguage{english} % switching to english
%%%%%%%%%%%%%%%%%%%%%%%%%%%%%%%%%%%%%%%%%%%%%%%%%%%%%%%%%%%%
\chapter*{Abstract}
\thispagestyle{empty}
%%%%%%%%%%%%%%%%%%%%%%%%%%%%%%%%%%%%%%%%%%%%%%%%%%%%%%%%%%%%

%\imiscomment{A short, english abstract of the work}
This project thesis displays our approach of planning and developing a simulation software for context aware systems. It covers both the aspects of the initial planning as well as the development phases itself.
This work originates from the course interaction- and media-development at the University of Lübeck.

The primary goal was to develop a simulator that interacts with the established MACK framework and its reasoners, without the need to build concrete hardware modules.
For the reason of reusability a generic simulation framework was designed, which focuses on modularity and abstraction, to enable  different user groups to simulate their own environments.

With this software (called CASi - context awareness simulator) we implemented a simulation that reflects the MATe application.

\vfill

%%%%%%%%%%%%%%%%%%%%%%%%%%%%%%%%%%%%%%%%%%%%%%%%%%%%%%%%%%%%
\section*{Keywords} context awareness, simulator, software development
%%%%%%%%%%%%%%%%%%%%%%%%%%%%%%%%%%%%%%%%%%%%%%%%%%%%%%%%%%%%
\selectlanguage{ngerman} % and back to german

% if using tocloft.sty, we need an explicit clearpage
\clearpage
% we want chapters and secitions in the toc, then generate it
\setcounter{tocdepth}{1}
\tableofcontents

% now we have real content, with section marks and egyptian numerals
\mainmatter

%%%%%%%%%%%%%%%%%%%%%%%%%%%%%%%%%%%%%%%%%%%%%%%%%%%%%%%%%%%%
\chapter{Einleitung}\label{chapter:introduction}
%%%%%%%%%%%%%%%%%%%%%%%%%%%%%%%%%%%%%%%%%%%%%%%%%%%%%%%%%%%%

%\imiscomment{Einführung und Motivation des Themas}

Die Arbeit beschreibt die Planung und Entwicklung eines Simulators zur realitätsnahen Context Awareness Simulation. Die Motivation für diese Projekt ist vor allem die Entwicklung einer Testumgebung, die die Analyse des MACK Frameworks ohne hohe Hardwarekosten ermöglicht.

%%%%%%%%%%%%%%%%%%%%%%%%%%%%%%%%%%%%%%%%%%%%%%%%%%%%%%%%%%%%
\section{Ziele der Arbeit}\label{sec:goals}
%%%%%%%%%%%%%%%%%%%%%%%%%%%%%%%%%%%%%%%%%%%%%%%%%%%%%%%%%%%%

%\imiscomment{Begründung für die Ziele und die Relevanz des Themas}
Das Ziel unserer Arbeit war vor allem die Entwicklung eines modularen Simulators, der sowohl flexibel erweitert werden, als auch universell eingesetzt, werden kann.

%%%%%%%%%%%%%%%%%%%%%%%%%%%%%%%%%%%%%%%%%%%%%%%%%%%%%%%%%%%%
\section{Stand der Technik}\label{sec:state_of_art}
%%%%%%%%%%%%%%%%%%%%%%%%%%%%%%%%%%%%%%%%%%%%%%%%%%%%%%%%%%%%

\imiscomment{Literaturrecherche und Darstellung anderer wichtiger Arbeiten zum Thema}

%\imiscomment{Darstellung des ``State of the Art''}
Zum Zeitpunkt der Entwicklung gab es bereits einen Context Awareness Simulator, der umfangreiche und konfigurierbare Simulationen ermöglichte. Dieser Simulator besitzt jedoch keinerlei Schnittstellen für die Integration von Sensoren und Aktuatoren. Aus diesem Grund haben wir uns für die Entwicklung eines eigenständigen Simulators entschieden, da die Weiterentwicklung und Anpassung des bestehenden Simulators einen sehr hohen, schwer kalkulierbare Aufwand bedeutet hätte.

%\imiscomment{Zitiert werden soll in der Arbeit wie in folgendem Beispiel:}

%Wie bereits mehrfach in der Literatur zu kontextualisierten, ambienten, erklärungsfähigen Systemen erläutert wurde \citep{Cassens-PhD-2008}, sind dabei vor allem Theorien und Modelle sozialer Interaktionen zwischen Mensch und Artefakt entscheidend. Auch \cite{Schmitt.ea-2010-mental_models_disappearing_systems} betrachten solche Systeme, hier vor allem zur Unterstützung der Kommunikation und Kollaboration in Teams. Ein wichtiger Aspekt ist auch die Intentionserkennung, die als neues Forschungsgebiet im Rahmen von Erklärungsfähigkeit und Ambienz begriffen werden kann. \citep{Kofod-Petersen.ea-2009-closed_doors} \gls{nemo}.

%Eine hervorragend Möglichkeit diese Gedanken auszudrücken bietet das Textsatzsystem \LaTeX\ mit seinen vielen ergänzenden Paketen \citepweb{ctan-cat}. Andere Themen finden sich im Web \citepweb{imis}.

%%%%%%%%%%%%%%%%%%%%%%%%%%%%%%%%%%%%%%%%%%%%%%%%%%%%%%%%%%%%
\section{Vorgehensweise}\label{sec:approach}
%%%%%%%%%%%%%%%%%%%%%%%%%%%%%%%%%%%%%%%%%%%%%%%%%%%%%%%%%%%%
Bei der Planung des Projektes haben wir zunächst eine Anforderungsanalyse durchgeführt. Hierfür haben wir Gespräche mit Felix Schmitt und Jörg Cassens, den Hauptinteressenten, geführt, um ein tiefergehendes Verständnis vom MACK Framework, der MATe Implementierung und den Anforderungen an einen Simulator zu erhalten.

Im Anschluss haben wir noch Gespräche mit Olof-Joachim Frahm und Daniel Wilken geführt um weitere technische Einblicke in das Framework und laufende Arbeiten zu erhalten.

Nach den Gesprächen haben wir bestehende Simulatoren getestet und recherchiert, ob bereits Arbeiten zum Thema durchgeführt wurden. Die Recherchen führten zu dem Entschluss, ein Simulationsframework zu entwickeln, dass flexibler als bestehende Simulatoren ist, da diese unter Berücksichtigung der Anforderungen ungeeignet waren.

Im Anschluss an die Recherchen haben wir die Architektur und eine Beschreibungssprachen auf XML Basis geplant und uns unabhängig von der Architektur textuell drei Szenerien (Büro, Krankenhaus, Museum) überlegt, mit denen wir die Vollständigkeit und Flexibilität der Architektur verifizieren konnten.

Die eigentliche Entwicklung ist feature-driven und user-centered. Aus diesem Grund haben wir zunächst den Simulationskern und dann die einzelnen Komponenten entwickelt und unserem Betreuer zu den wöchentlichen Treffen eine aktuelle Version als jar-Datei bereitgestellt.
%\imiscomment{Kurzer Überblick zur Vorgehensweise bei der Bearbeitung des Themas}



In Kapitel \ref{chapter:analysis} beschreiben wir die Analysen, die der Planung und Entwicklung des Simulators vorausgegangen sind.

Die Projektplanung sowie die Konzeption der Architektur werden im Kapitel \ref{chapter:concept} beschrieben.

Über die Entwicklung des Simulationsframeworks schreiben wir in Kapitel \ref{chapter:realization}.

Eine detaillierte Beschreibung der Simulation, mit der die MATe Anwendung getestet werden kann, findet sich im Kapitel \ref{chapter:dialogs}.

Abschließend beschreiben wir im Kapitel \ref{chapter:eval} mit welchen Methoden wir die Qualität der Software und das Erreichen unserer Ziele sichergestellt haben und bringen in Kapitel \ref{chapter:conclusions} eine Zusammenfassung unserer Arbeit sowie Ausblicke auf offene Punkte und Erweiterungsmöglichkeiten des Simulators, die als Thema für weitere Projekte dienen könnten.

%\imiscomment{Kurze Erläuterung, was in den einzelnen Kapitel dargestellt wird}


%%%%%%%%%%%%%%%%%%%%%%%%%%%%%%%%%%%%%%%%%%%%%%%%%%%%%%%%%%%%
\chapter{Analyse}\label{chapter:analysis}
In diesem Kapitel werden die Analyseschritte und die Vorüberlegungen beschrieben, die vor der Konzeption und Implementierung der Anwendung durchgeführt wurden. Hierzu gehen wir in Abschnitt \ref{sec:problem_ana} zunächst auf die Probleme und Einsatzszenarien ein und beschreiben dann in Abschnitt \ref{sec:user_ana} die Ergebnisse der Benutzeranalyse. Abschließend präsentieren wir in Abschnitt \ref{sec:context_ana} unsere Kontextanalyse.
%%%%%%%%%%%%%%%%%%%%%%%%%%%%%%%%%%%%%%%%%%%%%%%%%%%%%%%%%%%%

%\imiscomment{Analyse der Problemstellung, der Zielgruppen sowie des Anwendungskontextes}

%%%%%%%%%%%%%%%%%%%%%%%%%%%%%%%%%%%%%%%%%%%%%%%%%%%%%%%%%%%%
\section{Problemanalyse}\label{sec:problem_ana}
Bei der Entwicklung des MACK-Frameworks geht es vor allem um die Entwicklung eines zentralen Systems, das aus verschiedenen Reasonern besteht, die auf Basis von Sensordaten Zustände von Personen ermitteln können und diese an Aktuatoren weiterleiten. Dieses System soll durch verschiedene externe Komponenten wie Sensoren und Aktuatoren erweitert werden, um auf beliebige Kontexte adaptiert werden zu können. Hierbei entstehen komplexe Abhängigkeiten, da die ermittelten Werte der Sensoren vom Verhalten der Personen abhängen, welches wiederum zu einem gewissen Anteil von den Meldungen der Aktuatoren abhängen kann.

Um die Effizienz und Vollständigkeit der Reasoner zu verifizieren gibt es bislang nur die Möglichkeit, diese mit Hardware-Impementierungen der Sensoren und Aktuatoren zu verknüpfen und denkbare Szenarien durchzuführen. Dies eröffnet zum einen das Problem, dass zunächst die Hardware entwickelt werden muss, was mit hohem Zeit- und Kostenaufwand verbunden ist. Zum anderen muss das System von vielen Benutzern eingesetzt werden, um repräsentative Ergebnisse zu erhalten.

Darüber hinaus gibt es bisher unerforschte Einsatzgebiete für das MACK-Framework, für die es weder die Reasoner noch die notwendige Hardware gibt. Bevor die teure Hardware entwickelt oder eingekauft wird, soll getestet werden, ob das System für den angedachten Anwendungszweck geeignet ist.

Selbst mit, in ausreichendem Umfang vorhandener, Hardware ist es nur schwer möglich, die Wechselwirkung zwischen den Personen und dem Framework zu testen, da dies die Beobachtung der Benutzer in ihrem täglichen Handeln und eine entsprechende Verhaltensanalyse voraussetzt. Diese Analyse ist nur schwer zu realisieren und insbesondere dann sehr aufwändig, wenn die Auswirkungen minimaler Änderungen an Reasonern oder dem Framework an sich, verifiziert werden sollen.

Um eine Idee dafür zu bekommen, in welchen Einsatzgebiete das Framework eingesetzt werden könnte und welches demnach Gebiete sind, für die Simulationen realisierbar sein sollten, haben wir uns verschiedene Szenarien überlegt. Mit den nachfolgenden Szenarien haben wir im Verlauf der Planung und Implementierung verifizieren können, dass das Simulationsframework flexibel genug ist um auf verschiedene Einsatzgebiete adaptiert werden zu können.

Die Implementierung aller Szenarien im beschriebenen Umfang ist nicht Teil unseres Projektes. Wir beschränken uns hierbei auf eine grundlegende Implementierung des Büro-Szenarios in einer, ähnlich der im nachfolgenden beschriebenen, Form.

\subsection*{Szenario: Büro}

Das Büroszenario ist das Standardszenario des IMIS, für das bereits Teile einer Hardware-Implemen\-tierung und entsprechende Reasoner vorliegen.

Es gibt mehrere Büros, abgehend von einem Korridor. Jedes Büro hat einen Schreibtisch, der eine Dropzone\footnote{Bei der Dropzone handelt es sich um eine Station, in die verschiedene Benutzer Marken, bzw. ihre Schlüssel, legen können. Das Gerät erkennt dann, welche Marken vorhanden sind und führt so Rückschlüsse auf die im Raum befindlichen Personen durch.} und einen Cubus\footnote{Der Cubus ist ein Würfel, dessen Oberseite den vom Framework ermittelten Zustand des Besitzers angibt. Durch Drehen des Würfels kann eine explizite Änderung des Zustands erwirkt werden. Weitere Informationen befinden sich in Abschnitt 3.1 von \cite{doku-sensor-aktuator}.} beherbergt. Außerdem steht in jedem Büro ein Telefon und es gibt eine akustische Überwachung\footnote{Gemeint ist das in Abschnitt 3.5 von \cite{doku-sensor-aktuator} beschriebene \emph{Mike}.}, mit der festgestellt werden kann, ob der dort arbeitende Anwender gerade spricht. 

Es gibt an jeder Bürotür ein interaktives Türschild und eine Ampel, die Besuchern Informationen über die Unterbrechbarkeit des Bürobesitzers liefern.
Vom Flur abgehend gibt es Toiletten, eine Teeküche und einen Konferenzraum. Die Toiletten melden, wenn sie belegt sind, aus Gründen der Privatsphäre jedoch nicht, wer sie gerade benutzt. Personen haben die Möglichkeit, bevor sie aufstehen, zu sehen, ob die Toilette gerade besetzt ist. In der Teeküche gibt es einen Kaffeeautomaten, der gerne von Mitarbeitern zum Pausengespräch genutzt wird. Im Konferenzraum werden täglich die Tagesziele besprochen. Zu diesen Meetings müssen alle Mitarbeiter kommen.

Das Standardverhalten, wenn Personen keine spezifische Aktion vorhaben, ist, im Büro am Schreibtisch zu sitzen, sich Kaffee zu holen oder eine der Toiletten aufzusuchen.

Mögliche Zufallskomponente, die einen Anwender in seiner aktuellen Aktion unterbrechen könnten, sind zum Beispiel das Klingeln des Telefons oder andere Kommunikationsmittel. Wenn der Adressat im Radius des Kommunikationsmittels ist, beschäftigt er sich für eine zufällige Dauer mit der Kommunikation und unterbricht dafür seine Arbeit.

\subsection*{Szenario: Museum}
 
In einem Museum gibt es verschiedene Bereiche mit unterschiedlichen Themengebieten, zum Beispiel könnte es in einer Ausstellung über die menschlichen Sinne die Bereiche \glqq Sehen\grqq , \glqq Hören\grqq\ und \glqq Fühlen\grqq. Außerdem gibt es Toiletten und einen Souvenirladen.

Im Museum halten sich Besucher (20 – 100) und verschiedene Mitarbeiter (10 – 20) auf. Einige Mitarbeiter machen Führungen mit Besuchern durch bestimmte Bereiche, einige Mitarbeiter bewachen empfindliche oder wertvolle Ausstellungsstücke und können Besuchern Wege zu bestimmten Bereichen wie Toiletten und dem Ausgang weisen.

Am Ein- und Ausgang befindet sich ein Lageplan für das Museum, an verschiedenen Ausstellungsstücken gibt es Informationstafel und beim Eintritt in das Museum können die Besucher einen elektronischen Führer mitnehmen. Der elektronische Führer ist Sensor und Aktuator zugleich. Er ermöglicht dem Besucher, sich die Hinweise auf den Informationstafeln vorlesen zu lassen und merkt sich, welche Ausstellungsstücke und Bereiche des Museums den Besucher interessieren. Auf Grundlage dieser Daten macht er Vorschläge, welche anderen Bereiche und Ausstellungsstücke interessant für diesen Besucher sein könnten. Außerdem schlägt er Führungen durch das Museum vor, die ebenfalls auf die Interessen des Benutzers angepasst sind. Der elektronische Führer könnte zusätzlich beim Betreten des Souvenirladens anzeigen, welche Souvenirs dem Besucher gefallen könnten.

Es gibt noch weitere Sensoren innerhalb der Bereiche, die feststellen können, wie viele Personen sich dort aufhalten. Durch Aktuatoren an den, in diese Bereiche führenden, Türen könnte den Besuchern angezeigt werden, welche Bereiche gerade überfüllt oder wenig besucht sind.

\subsection*{Szenario: Krankenhaus}

In einem Krankenhaus auf einer Intensivstation arbeiten zehn Krankenschwestern und zwei Ärzte. Es gibt zehn Einzelzimmer und zehn Doppelzimmer. Insgesamt gibt es 30 Patienten. Die zehn Patienten in den Einzelzimmern sind in einem kritischen Zustand und deshalb an ein Health-Monitoring-System (HMS) angeschlossen, welches den Sauerstoffgehalt im Blut, den Blutdruck, den Puls und die Atemfrequenz erfasst. Außerdem stuft das System den Patienten anhand dieser Werte in Kategorien ein: \glqq sehr gut\grqq, \glqq gut\grqq, \glqq stabil\grqq, \glqq kritisch\grqq, \glqq lebensgefährlich\grqq\ und \glqq unbekannt\grqq.

Alle Mitarbeiter verfügen über einen Tablett PC, über den sie innerhalb der Station lokalisiert werden können. Darüber hinaus werden die Geräte zur Protokollierung der Visiten verwendet. Wenn ein Patient in den Zustand \glqq kritisch\grqq\ oder schlechter wechselt, werden beide Ärzte über das Tablett informiert. Sie erhalten die Zustandseinschätzung des HMS, die Analysewerte und die Information, welcher Patient in welchem Raum betroffen ist.

Die Station hat außerdem zwei Computerarbeitsplätze im Schwesternzimmer, sowie zwei, mit Computern ausgestattete, Behandlungszimmer, in denen sich gleichzeitig die Schreibtischarbeitsplätze der Ärzte befinden. Alle Computer sind mit einem \glqq Desktop Activity Analyzer\footnote{Der DAA überwachte die Aktivität und die Programmart, die sich im Vordergrund befindet. Siehe Abschnitt 3.4 von \cite{doku-sensor-aktuator}.}\grqq\ ausgestattet.

Die Simulation beginnt am 1.11.2011 um 4:30. Eine der beiden Nachtschwestern macht einen Kontrollgang. Danach begibt sie sich zurück ins Schwesternzimmer und arbeitet für zwei Stunden am PC. Um 5:00 Uhr wechselt das HMS in Raum 1 von \glqq stabil\grqq\ auf \glqq kritisch\grqq. Wenn das Panel im Schwesternzimmer dies anzeigt, soll eine Schwester sofort in das entsprechende Zimmer gehen und Gegenmaßnahmen vornehmen, so dass der Zustand im Anschluss mindestens \glqq stabil\grqq\ ist. Danach setzt die Schwester ihr vorherige Aktivität fort. 

Um 5:30 Uhr macht die andere Schwester einen Rundgang. Um 6:00 Uhr kommt der erste Arzt. Er holt sich zunächst einen Kaffee aus der Küche und geht danach in sein Zimmer  um dort für 30 Minuten am Computer die Vorkommnisse der letzten Nacht durchzugehen (lesen). Der zweite Arzt kommt um 6:15 Uhr, holt sich ebenfalls einen Kaffee und geht danach in sein Zimmer um an seinem Computer die Vorkommnisse der letzten Nacht zu analysieren. Um 6:45 Uhr kommen die ersten fünf Tagesschwestern. Um 7:00 Uhr treffen sich alle anwesenden Mitarbeiter zur Besprechung im Schwesternzimmer (1 Stunde). Danach gehen die Nachtschwestern nach Hause. Um 9:00 wird der Patient aus Zimmer 3 in den OP gebracht. Arzt 1 operiert für 3 Stunden zusammen mit 5 Schwestern, die dafür zur Arbeit kommen, den Patienten. Während dieser Zeit ignoriert er alle Meldungen, d.h. er unterbricht diese Aktion nicht, selbst wenn das System einen Vorfall bei einem anderen Patienten meldet.

Um 10:00 Uhr melden die HMS in R4 und R5, dass der Zustand von \glqq gut\grqq\ auf \glqq lebensgefährlich\grqq\ wechselt. Der freie Arzt unterbricht seine Aktivität und geht in das erste Zimmer. Die Schwester, die am nächsten ist, geht in das zweite Zimmer. Dem operierenden Arzt wird die Situation via Tablett Computer gemeldet. Der freie Arzt erkennt eine Fehlfunktion des Systems und setzt den Zustand wieder auf \glqq gut\grqq. Anschließend geht er zum zweiten kritischen Patienten. 

Um 11:00 Uhr beginnen der Arzt und zwei Schwestern mit der Visite. Währenddessen meldet sein Smartphone eine kritische Situation im OP, so dass er zunächst in den OP geht, um dem operierenden Arzt zu assistieren. Nach einer Stunde sind beide Ärzte mit der Operation fertig und gehen den Aktionen nach, die sie während der Operation ignoriert haben, sofern diese nicht bereits von anderen erledigt wurden.

Jeder Arzt hat eine Schwester, die stündlich zu ihm kommen muss, um Aktuelles mit ihm zu besprechen. Bevor sie dies tun, prüfen sie mit ihrem Tablett, ob der jeweilige Arzt verfügbar ist und wo er sich befindet. Das System liefert hierzu die Raumnummer und den Status des Arztes (\glqq Unterbrechbar\grqq, \glqq Nicht unterbrechbar\grqq\ oder \glqq Unbekannt\grqq).
%%%%%%%%%%%%%%%%%%%%%%%%%%%%%%%%%%%%%%%%%%%%%%%%%%%%%%%%%%%%

%\imiscomment{Problemanalyse oder Aufgabenanalyse (Aufgaben, die mit dem System von einem Benutzer gelösten werden können sollen)}

% The following is an example of an IMIS-styled table.
% You are free to use other styles, in fact, encouraged.
% You should look at other options, e.g. using booktable.sty
% Bringhurst says on the issue of rules etc. in tables:
%   There should be a minimum amount of furniture (rules,
%   boxes, dots and other guiderails for traveling through
%   typographic space) and a maximum amount of information.
% You cannot follow Bringhurst and IMIS at the same time,
% your choice.
%\begin{table}[ht]
% \newcolumntype{Y}{>{\columncolor[gray]{.85}}>{\raggedright\arraybackslash}X}
%  \begin{tabularx}{\columnwidth}{!{\vrule width 1pt}Y!{\vrule width 1pt}l|l|l|l|l|l|l!{\vrule width 1pt}}
%    \noalign{\hrule height 1pt}
%    \rowcolor[gray]{.85}
%    \textbf{bis} & \textbf{15.5.} & \textbf{22.5.} & \textbf{5.6.} & \textbf{12.6.} & \textbf{19.6.} & \textbf{26.6.} & \textbf{3.7.} \\\noalign{\hrule height 1pt}
%    \textbf{Content Theory}  & Lübeck & & Kiel     & Lübeck   & Final        &       & \\\hline
%    \textbf{Content Videos}  & Lübeck & & Kiel     & Kiel     & Final        &       & \\\hline
%    \textbf{Cover, Icon}     & Kiel   & & Lübeck   & Kiel     & Kiel         & Final & \\\hline
%    \textbf{Videobearbeitung}& Lübeck & & SLH, LOS & BNR, IGS & CFW,CJB      & Final & \\\hline
%    \textbf{Aufbereitung der Videos}
%                             & Kiel   & & Kiel     & Lübeck   & Lübeck       & Final & \\\hline
%    \textbf{Bilder/Videos der Interviews}
%                             & Lübeck & & Lübeck   & Kiel     & Kiel/Lübeck  & Final & \\\hline
%    \textbf{Audio-Sprechertexte}
%                             & Lübeck & & Kiel     & Kiel     & Lübeck       & Final & \\
%    \noalign{\hrule height 1pt}
%  \end{tabularx}
%  \caption{Beispieltabelle}
%  \label{table:example}
%\end{table}%
%
%\begin{table}[ht]
%  \begin{center}
%    \begin{tabular}{lrr}
%      \textbf{Beruf}  & \textbf{Anzahl} & \textbf{Anteil}\\ \hline
%      Student/in         & 17 & 65.38\% \\
%      Wissenschaftler/in &  3 & 11.54\% \\
%      Professor/in       &  2 &  7.69\% \\
%      Forscher/in        &  1 &  3.85\% \\
%      Informatiker/in    &  1 &  3.85\% \\
%      Führungskraft      &  1 &  3.85\% \\
%      Verkaufskraft      &  1 &  3.85\% \\
%    \end{tabular}
%    \caption{Berufe der Umfrageteilnehmer}
%    \label{table:berufe}
%  \end{center}
%\end{table}%

%%%%%%%%%%%%%%%%%%%%%%%%%%%%%%%%%%%%%%%%%%%%%%%%%%%%%%%%%%%%
\section{Benutzeranalyse}\label{sec:user_ana}

Bei der Zielgruppe handelt es sich um erfahrene Entwickler, die mit dem Simulator ihre Anwendungen im Bereich Context Awareness testen möchten und die Anwendung als Werkzeug bei ihrer Arbeit einsetzen. In erster Linie sind das die Mitarbeiter des IMIS, die das MACK-Framework entwickeln und gleichzeitig die Auftraggeber dieses Projektes sind. Sie wollen bestimmte Teile des MACK-Systems wie zum Beispiel seine Reasoner testen und evaluieren.

Da wir die späteren Benutzer als Experten eingestuft haben, standen wir von Anfang an viel mit ihnen in Kontakt, um beispielsweise die spätere Spezifikationssprache der Simulationen festzulegen. Unser Ergebnis war hier, dass mindestens grundlegende Kenntnisse in Java und XML gegeben sind und deshalb die Simulationen mittels dieser Sprachen spezifiziert werden können. Die Erstellung der Simulation nur anhand von Java-Klassen ist zwar umständlich, aber so konnten wir den Fokus auf die Entwicklung des eigentlichen Simulators setzen.

Um eine Simulation zu beschreiben, sollen folgende Teile definiert werden können:
\begin{itemize}
\item Eine Umgebung beispielsweise ein Gebäude mit Räumen und Türen.
\item Personen, die an der Simulation beteiligt sind, diese werden im Weiteren Agents genannt.
\item Verschiedene Sensoren und Aktuatoren, die beteiligt sind und getestet werden sollen.
\item Aktionen, die von Agents oder Sensoren/Aktuatoren angestoßen werden, um reale Abläufe zu simulieren.
\end{itemize}
  

Eine weitere wichtige Frage an die erfahrenen Benutzer war, welche Art von Benutzungsschnittstellen für sie implementiert werden sollten. Wir haben erfahren, dass sowohl besonderer Wert auf detaillierte Logausgaben gelegt wird als auch eine einfache grafische Oberfläche erwünscht ist, da diese für das Erlangen eines ersten Eindrucks von der Arbeit des Systems besonders gut geeignet ist. Die grafische Oberfläche soll hierbei ausschließlich zur Visualisierung und nicht zum Erstellen der Simulation dienen. Ansonsten  wäre der Aufwand der Implementierung zu groß gewesen. Auf den Ablauf der Simulation soll nur sehr begrenzt Einfluss genommen werden können, zum Beispiel durch Verändern der Simulationsgeschwindigkeit oder Pausieren der Simulation.

Im gesamten Entwicklungsprozess haben wir eng mit den späteren Nutzern zusammengearbeitet und uns möglichst früh Rückmeldung zu offenen Fragen und Entwicklungsentscheidungen geholt. So wollten wir sicherstellen, dass die Benutzer direkt an der Entwicklung beteiligt sind und somit mögliche Fehlentwicklungen vermieden werden.
%%%%%%%%%%%%%%%%%%%%%%%%%%%%%%%%%%%%%%%%%%%%%%%%%%%%%%%%%%%%

%\imiscomment{Zielgruppen}

%%%%%%%%%%%%%%%%%%%%%%%%%%%%%%%%%%%%%%%%%%%%%%%%%%%%%%%%%%%%
\section{Kontextanalyse}\label{sec:context_ana}
Der Simulator soll im IMIS auf den Arbeitsplätzen der Entwickler laufen. Die Arbeitsplätze sind mindestens Quad-Core-Computer. Der Simulator soll während der Entwicklung zur Verifikation einzelner Änderungen eingesetzt werden. Außerdem sollen längere Simulationen durchgeführt werden, deren Ergebnisse später mit Hilfe der Logfiles ausgewertet werden können. Das System muss also sowohl im Vorder- als auch im Hintergrund laufen können.

Wegen des hohen Netzwerkaufkommens ist es optimal, Server und Simulator auf einer Maschine laufen lassen zu können und den Netzwerkverkehr über lokale Loopback-Devices abzuwickeln.
%%%%%%%%%%%%%%%%%%%%%%%%%%%%%%%%%%%%%%%%%%%%%%%%%%%%%%%%%%%%
%%%%%%%%%%%%%%%%%%%%%%%%%%%%%%%%%%%%%%%%%%%%%%%%%%%%%%%%%%%%
\chapter{Konzeption}\label{chapter:concept}
%%%%%%%%%%%%%%%%%%%%%%%%%%%%%%%%%%%%%%%%%%%%%%%%%%%%%%%%%%%%

%\imiscomment{Grobkonzeption der Arbeit}

%\imiscomment{Keine Codedarstellung, allenfalls Pseudocode}

%\imiscomment{Struktur dieses Kapitel kann je nach Problemstellung unterschiedlich gestaltet werden}

%%%%%%%%%%%%%%%%%%%%%%%%%%%%%%%%%%%%%%%%%%%%%%%%%%%%%%%%%%%%
\section{Systemarchitektur}\label{sec:architecture}
%%%%%%%%%%%%%%%%%%%%%%%%%%%%%%%%%%%%%%%%%%%%%%%%%%%%%%%%%%%%

In Abbildung \ref{fig:architecture} wird die Modularität der Architektur des Simulators deutlich. Die Komponenten in den weißen Kästen stellen Module dar, die durch den Austausch weniger Codezeilen ausgewechselt werden können. Diese Komponenten sind zum Teil simulationsspezifisch. Zum Beispiel sind Sensoren, Aktuatoren und Aktionen vom Simulationsumfeld abhängig. Darüber hinaus bestehen auch die Abhängigkeiten Sensoren $\leftrightarrow$ Aktionen und Aktuatoren $\leftrightarrow$ Aktionen, da Sensoren auf bestimmte Aktionen reagieren und Aktuatoren bestimmte Aktionen auslösen können.

Die Pfeile in der Abbildung geben den Hauptinformationsfluss zwischen den Komponenten an.

% this is an example for using TikZ:
\begin{figure}[htb]
  \begin{center}
	\begin{tikzpicture}[minimum width=3.3cm,minimum height=0.8cm]
	\tikzset{
    myarrow/.style={->, >=latex', shorten >=1pt, thick},
	highlight/.style={rectangle, draw, fill=black!10},
	normal/.style={rectangle,rounded corners, draw}
} 
	\node[highlight] (engine) at (2,-2) [draw] {\emph{Simulationsengine}};
	\node[normal] (generator) [above left=of engine]{Generatoren};
	\node[normal] (kommunikation) [right=of engine]{Kommunikation};
	\node[normal] (benutzung) [below left=of engine]{Benutzung};
	\node[normal] (sensoren) [above right=of engine]{Sensoren};
	\node[normal] (aktuatoren) [below right=of engine]{Aktuatoren};
	\node[normal] (aktionen) [above=of engine]{Aktionen};
	\node[normal] (aktionsverwaltung) [below=of engine]{Aktionsverwaltung};
	
	\draw[->, thick, >=latex', shorten >=1pt]
	(generator) edge (engine)
	(aktuatoren) edge (engine)
	(engine) edge (sensoren)
	(kommunikation) edge (aktuatoren)
	(sensoren) edge (kommunikation)
	;
	\draw[<->, thick, >=latex', shorten >=1pt]
	(engine) edge (kommunikation)
	(engine) edge (aktionen)
	(engine) edge (benutzung)
	(engine) edge (aktionsverwaltung)
	;
    \end{tikzpicture}
  \end{center}
  \caption{{Unterteilung der Simulatorarchitektur in Module}}
  \label{fig:architecture}
\end{figure}

%%%%%%%%%%%%%%%%%%%%%%%%%%%%%%%%%%%%%%%%%%%%%%%%%%%%%%%%%%%%
\subsection{Simulationsengine}\label{subsec:concept_engine}

Der Kern des Simulationsframeworks ist die Simulationsengine, die im wesentlichen aus den Klasse \codeclass{SimulationEngine} und \codeclass{SimulationClock} besteht. Sowohl die Clock als auch die Engine sind als Singleton realisiert und können deshalb aus jeder Klasse angesprochen werden.

Die SimulationClock repräsentiert die Zeit in der Simulation. Die Geschwindigkeit der Uhr und somit die Geschwindigkeit der Simulation kann skaliert werden. Der Skalierungsfaktor in der Uhr gibt dabei an, wie viele Millisekunden in Echtzeit einer simulierten Sekunde entsprechen. Somit resultiert ein niedriger Wert in einer höheren Geschwindigkeit.

Klassen können das \codeclass{ISimulationClockListener}-Interface implementieren und sich bei der Clock als Listener registrieren, um über Events wie das Ticken der Uhr, das Pausieren, Starten und Stoppen der Simulation informiert zu werden.

Die Engine hält die Welt (\codeclass{World}) bereit, welche die Konfiguration und das Verhalten der Simulation beschreibt. Darüber hinaus hält die Engine Referenzen auf den Kommunikationshandler, der in unserer Implementierung für die Kommunikation mit dem Awareness-Hub zuständig ist.
%%%%%%%%%%%%%%%%%%%%%%%%%%%%%%%%%%%%%%%%%%%%%%%%%%%%%%%%%%%%
%\imiscomment{Grobkonzeption eines einzelnen Moduls}

%%%%%%%%%%%%%%%%%%%%%%%%%%%%%%%%%%%%%%%%%%%%%%%%%%%%%%%%%%%%
\subsection{Generatoren}\label{subsec:concept_generators}
Generatoren erzeugen \codeclass{World}-Objekte, die von der Simulationsengine simuliert werden können.

Ein Generator muss sich darum kümmern, Räume, Sensoren, Aktuatoren, Agenten und Aktionen zur erzeugen und geeignet zu verknüpfen. In unserer Implementierung gibt es einen Generator, der eine Welt erzeugt, in dem fest einprogrammiert Java-Objekte erzeugt werden.

Eine weitere Möglichkeit für einen Generator wäre ein XML Handler, der Java-Objekte aus einer XML-Beschreibung generiert.

Generatoren können in der Main-Klasse des Simulators ausgetauscht werden. Die einzigen Vorraussetzungen, die ein Generator erfüllen muss, sind, dass er das \codeinterface{IWorldGenerator}-Interface implementiert und eine vollständige Welt erzeugt.
%%%%%%%%%%%%%%%%%%%%%%%%%%%%%%%%%%%%%%%%%%%%%%%%%%%%%%%%%%%%


%%%%%%%%%%%%%%%%%%%%%%%%%%%%%%%%%%%%%%%%%%%%%%%%%%%%%%%%%%%%
\subsection{Benutzungsschnittstellen}\label{subsec:concept_interfaces}
Benutzungsschnittstellen können bereit gestellt werden, indem eine Implementierung des Interfaces \codeinterface{IMainView} in der Main-Klasse des Simulators an den \codeclass{MainController} übergeben wird. Die Benutzungsschnittstellen können sowohl passiv als auch interaktiv gestaltet werden. Hierfür können sich Objekte des Interfaces als Listener bei den Objekten des Modells registrieren und auf die \codeclass{SimulationClock} und die Engine zugreifen, da beide Komponenten als Singleton realisiert sind.
%%%%%%%%%%%%%%%%%%%%%%%%%%%%%%%%%%%%%%%%%%%%%%%%%%%%%%%%%%%%

%%%%%%%%%%%%%%%%%%%%%%%%%%%%%%%%%%%%%%%%%%%%%%%%%%%%%%%%%%%%
\subsection{Aktionen}\label{subsec:concept_actions}
Aktionen können allgemein durch folgende Parameter spezifiziert werden:
\begin{description}
	\item[priority] Die Priorität einer Aktion als Integer zwischen \texttt{0} und \texttt{10}.
	\item[duration] Die Dauer in Sekunden oder \texttt{-1}, falls keine Dauer angegeben wird (z.B. bei \texttt{Move}-Aktionen, die dann fertig sind, wenn das Ziel erreicht wurde).
	\item[earliestStartTime] ein Zeitpunkt (\codeclass{SimulationTime}), ab dem die Aktion gestartet werden darf oder \codeinline{null}, falls die Aktion zu einem beliebigen Zeitpunkt gestartet werden kann.
	\item[deadline] ein Zeitpunkt, zu dem die Aktion erledigt sein muss oder \codeinline{null}, falls keine Headline angegeben wurde.
	\item[state] der aktuelle Status der Aktion, z.B. \codeinline{SCHEDULED}, wenn die Aktion einem Agenten hinzugefügt wurde, dieser sie aber noch nicht ausgeführt hat, \codeinline{ONGOING}, wenn die Aktion gerade durchgeführt wird oder \codeinline{COMPLETED}, wenn die Aktion erfolgreich durchgeführt wurde.
\end{description}
Es gibt \codeclass{AtomicAction}s und \codeclass{ComplexAction}s, welche aus einer Liste von atomaren Aktionen bestehen können. Mit diesem Konstrukt lassen sich unter anderem Aktionen wie das aufsuchen und reden mit einer anderen Person (\codeclass{Agent}) beschreiben.

Um eine neue Aktion zur Verfügung zu stellen, muss eine neue Klasse implementiert werden, die von einer der abstrakten Klassen \codeclass{AtomicAction} oder \codeclass{ComplexAction} erbt. Die eigentliche Aktion muss in der \codemethod{internalPerform(AbstractComponent performer)}-Methode beschrieben werden. Sollten vor der ersten Ausführung der Aktion Konfigurationsschritte notwendig sein, können diese in der Methode \codemethod{preActionTask(AbstractComponent performer)} beschrieben werden. Analog dazu gibt es auch eine Methode \codemethod{postActionTask(AbstractComponent performer)}, die vom Framework aufgerufen wird, nachdem der Job erledigt wurde.

Wenn eine Dauer angegeben wurde, kümmert sich dass Simulationsframework um das Dekrementieren der Zeit, in diesem Fall muss die \codemethod{internalPerform(AbstractComponent per\-for\-mer)}-Methode immer \codeinline{false} zurückgeben. Sobald diese Methode \codeinline{true} zurückgibt, gilt die Aktion als erledigt und es wird im Falle einer komplexen Aktion mit der nächsten atomaren Aktion fortgefahren.

Aktionen müssen vom Generator erzeugt werden und einem oder mehreren Agenten hinzugefügt werden.
%%%%%%%%%%%%%%%%%%%%%%%%%%%%%%%%%%%%%%%%%%%%%%%%%%%%%%%%%%%%

%%%%%%%%%%%%%%%%%%%%%%%%%%%%%%%%%%%%%%%%%%%%%%%%%%%%%%%%%%%%
\subsection{Aktionsverwaltung}\label{subsec:concept_actionhandling}
Zur Aktionsverwaltung hat jeder Agent einen \codeinterface{IActionScheduler}, der im Konstruktor ausgetauscht werden kann. Die Aufgabe des Schedulers ist es, die Aktions-Sammlungen bereit zu halten. Das Konzept sieht vor, dass es für jeden Agenten drei Listen von Aktionen gibt:
\begin{description}
	\item[todoList] Diese Liste beinhaltet Aufgaben, die in jedem Fall während der Simulation entsprechend ihrer oben beschriebenen Parameter ausgeführt werden sollen.
	\item[actionPool] Diese Menge enthält Aufgaben, die optional ausgeführt werden können, wenn der Agent gerade keine anderen Aufgaben zu erledigen hat. Wenn eine Aufgabe aus dem Pool abgeschlossen wird, wird diese nicht aus dem Pool entfern. So können sich wiederholenden Zufallsaktionen simuliert werden.
	\item[interruptActionList] In diese Liste werden während der Simulation Aktionen eingeordnet, die in jedem Fall unmittelbar als nächste Aktionen ausgeführt werden müssen. Sinnvoll ist dies insbesondere dann, wenn ein Agent darauf angewiesen ist, dass ein anderer Agent mit ihm interagiert. Dem zweiten Agenten kann dann eine Aktion auf die Interuppt-Liste gesetzt werden, damit er unmittelbar bei der nächsten Gelegenheit mit dem ersten Agenten interagiert und keine anderen Aktionen vorzieht.
\end{description}

Die Kernfunktionalität eines Action-Schedulers ist es, mit geeigneten Algorithmen zu jedem Zeitpunkt eine Aktion auszuwählen. Hierzu ruft der Agent die \codemethod{getNextAction()}-Methode auf, worauf hin der Scheduler eine Aktion auswählt und diese von der Liste löscht.
%%%%%%%%%%%%%%%%%%%%%%%%%%%%%%%%%%%%%%%%%%%%%%%%%%%%%%%%%%%%


%%%%%%%%%%%%%%%%%%%%%%%%%%%%%%%%%%%%%%%%%%%%%%%%%%%%%%%%%%%%
\subsection{Kommunikationshandlern}\label{subsec:concept_communication}
Ein Kommunikationshandler muss das \codeinterface{ICommunicationHandler}-Interface implementieren. Kommunikationshandler sind für die Verwaltung der Kommunikation mit einem beliebigem Gegenpart außerhalb des Simulators zuständig. Ein konkretes Beispiel ist ein Netzwerkhandler, der die Kommunikation über eine Netzwerkschnittstelle verwalten kann. In der aktuellen Implementierung ist der \codeclass{MACKNetworkHandler} ein Kommunikationshandler der den Informationsaustausch mit dem MACK Server realisiert.

Innerhalb der Simulation interagiert der Kommunikationshandler mit Sensoren und Aktuatoren, die beide das Interface \codeinterface{ICommunicationComponent} implementieren.. Die Komponenten registrieren sich in der Initialisierungsphase des Simulators beim Handler, indem die Methode \codemethod{re\-gister (ICom\-muni\-cation\-Component comp)} aufgerufen wird.

Während der Simulation können die Komponenten die \codemethod{send(ICommunication\-Component\\
sender, Object message)}-Methode aufrufen, um eine Nachricht an den Handler zu senden. Umgekehrt kann der Handler bei den Komponenten die Methode \codemethod{receive(Object message)} aufrufen, um eine neu eingetroffene Nachricht zu übermitteln.

Zu beachten ist, dass die Nachrichten zunächst vom Typ \codeclass{Object} sind, um einen hohen Abstraktionsgrad zu gewährleisten. Von welchem Typ die Nachrichten wirklich sind, muss in Sensoren, Aktuatoren und dem Handler während der Implementierung entschieden werden.

In der spezifischen Simulation und dem \codeclass{MACKCommunicationHandler} sind Nachrichten immer vom Typ \codeclass{String}, wobei die Nachrichten im XML-Format sind, welches auch das MACK-Framework verwendet. Ausgehende Nachrichten werden von den Sensoren und Aktuatoren unter Verwendung der \codeclass{MACKProtocolFactory} erzeugt.
%%%%%%%%%%%%%%%%%%%%%%%%%%%%%%%%%%%%%%%%%%%%%%%%%%%%%%%%%%%%


%%%%%%%%%%%%%%%%%%%%%%%%%%%%%%%%%%%%%%%%%%%%%%%%%%%%%%%%%%%%
\subsection{Sensoren und Aktuatoren}\label{subsec:concept_actuators_sensors}
Da Sensoren und Aktuatoren viele Gemeinsamkeiten aufweisen, erben alle konkreten Sensoren und Aktuatoren von der abstrakten Klasse \codeclass{AbstractInteractionComponent}, die auch das \codeinterface{ICom\-muni\-cation\-Component}-Interface implementiert.

Beide Komponenten haben einen Bereich, in dem sie auf Aktivitäten reagieren oder bei den Agenten Aktivitäten auslösen können. Der Bereich kann als Ausrichtung (Blickrichtung), Öffnungswinkel und Radius definiert werden, indem der entsprechende Konstruktor verwendet wird. Werden diese Parameter nicht angegeben, wird der komplette Raum, in dem sich eine Komponente befindet überwacht. Mit der Methode \codemethod{setShapeRepresentation(Shape shape)} kann darüber hinaus eine beliebige Shape als Überwachungsbereich festgelegt werden.

Die Komponenten können die Methoden \codemethod{boolean checkInterest\-(AbstractAction\\
action, Agent agent)} und \codemethod{boolean checkInterest(Agent agent)} implementieren, um anzugeben ob die jeweilige Komponente an einem Agenten und/oder einer Aktion eines Agentens interessiert ist. Standardmäßig geben beide Methoden \codeinline{false} zurück. In der Initialisierungsphase werden alle interessierten Komponenten den Agenten als \codeinterface{IExtendedAgentListener} hinzugefügt und werden damit automatisch informiert, wenn ein Agent, an dem die Komponente interessiert ist, eine Aktion durchführt.

 Bereits in der Elternklasse \codeclass{AbstractInteractionComponent} aller Sensoren und Aktuatoren wird darauf geachtet, dass die eigentliche Methode zum Reagieren auf Agenten und Aktionen (\codemethod{boolean handleInternal(AbstractAction action, Agent agent)}) nur dann aufgerufen wird, wenn sich der Agent im Überwachungsbereich der Komponente befindet.

Der einzige Unterschied zwischen Sensoren und Aktuatoren ist, dass der Rückgabewert der \codemethod{handle\-Internal}-Methode von Sensoren ignoriert wird. Bei Aktuatoren oder Mischformen von Sensoren und Aktuatoren ist der Rückgabewert als Erlaubnis für eine Aktion zu verstehen. Wenn also ein Agent eine Aktion durchführt und die Methode \codeinline{false} zurückgibt, bricht der Agent die Aktion ab.

Der Typ der Komponenten wird angeben, indem die \codeinline{type}-Variable gesetzt wird. Sie kann die Werte \codeinline{SENSOR}, \codeinline{ACTUATOR} und \codeinline{MIXED} annehmen. Der Typ \codeinline{MIXED} kann unter anderem für Wearables wie z.B. Smartphones sinnvoll sein.

Wearables können mit dem Konzept realisiert werden, indem mit der Methode \codemethod{setAgent(Agent agent)} ein Agent gesetzt wird, der der Hauptinhaber der Komponente ist. Darüber hinaus muss mit der Methode \codemethod{setWearable(boolean wearable)} festgelegt werden, dass die Komponente ein Wearable ist. Dann wird jedes mal, wenn der Agent seine Position ändert, die Position der Komponente ebenfalls neu gesetzt.
%%%%%%%%%%%%%%%%%%%%%%%%%%%%%%%%%%%%%%%%%%%%%%%%%%%%%%%%%%%%

%%%%%%%%%%%%%%%%%%%%%%%%%%%%%%%%%%%%%%%%%%%%%%%%%%%%%%%%%%%%
\chapter{Realisierung}\label{chapter:realization}
%%%%%%%%%%%%%%%%%%%%%%%%%%%%%%%%%%%%%%%%%%%%%%%%%%%%%%%%%%%%

\imiscomment{Beschreibung der hard- und software-technischen Realisierung}

\imiscomment{Struktur dieses Kapitel kann je nach Problemstellung unterschiedlich gestaltet werden}

%%%%%%%%%%%%%%%%%%%%%%%%%%%%%%%%%%%%%%%%%%%%%%%%%%%%%%%%%%%%
\section{Realisierung der Einzelmodule}
%%%%%%%%%%%%%%%%%%%%%%%%%%%%%%%%%%%%%%%%%%%%%%%%%%%%%%%%%%%%

%%%%%%%%%%%%%%%%%%%%%%%%%%%%%%%%%%%%%%%%%%%%%%%%%%%%%%%%%%%%
\subsection{Realisierung von Modul 1}\label{subsec:real_module_1}02-M-MCI-History-JC-master
%%%%%%%%%%%%%%%%%%%%%%%%%%%%%%%%%%%%%%%%%%%%%%%%%%%%%%%%%%%%

\imiscomment{UML-Diagramme}

\imiscomment{Schnittstellen}

\imiscomment{Datenmodelle}

You might want to include listings:

% this example listing is inside a float environment, but
% the option [H] makes it stay put:
\begin{illfloat}[H]
  \begin{lstlisting}
for i:=maxint to 0 do
begin
{ do nothing }
end;
Write('Case insensitive ');
Write('Case insensitive ');
Write('Case insensitive ');
Write('Case insensitive ');
Write('Case insensitive ');
Write('Case insensitive ');
WritE('Pascal keywords.');
  \end{lstlisting}%
  \illcaption{Ein Quelltext -- wirklich hier.}
\end{illfloat}

%%%%%%%%%%%%%%%%%%%%%%%%%%%%%%%%%%%%%%%%%%%%%%%%%%%%%%%%%%%%
\subsection{Realisierung von Modul 2}\label{subsec:real_module_2}
%%%%%%%%%%%%%%%%%%%%%%%%%%%%%%%%%%%%%%%%%%%%%%%%%%%%%%%%%%%%

% this example listing can float - we can not use the
% listings.sty option[flat] since that would disconnect
% the listing from the illcaption
\begin{illfloat}
  \begin{lstlisting}
for i:=maxint to 0 do
begin
{ do nothing }
end;
Write('Case insensitive ');
Write('Case insensitive ');
Write('Case insensitive ');
Write('Case insensitive ');
Write('Case insensitive ');
Write('Case insensitive ');
Write('Case insensitive ');
Write('Case insensitive ');
Write('Case insensitive ');
WritE('Pascal keywords.');
  \end{lstlisting}%
  \illcaption{Test einer sehr langen Beschreibung um den Flow zu testen. Test einer sehr langen Beschreibung um den Flow zu testen. Test einer sehr langen Beschreibung um den Flow zu testen. Test einer sehr langen Beschreibung um den Flow zu testen. Test einer sehr langen Beschreibung um den Flow zu testen. Test einer sehr langen Beschreibung um den Flow zu testen.}[Optionaler Kurzeintrag]
\end{illfloat}

You do not want to show full program listings, but short snipplets to show how a protocol looks like.

%%%%%%%%%%%%%%%%%%%%%%%%%%%%%%%%%%%%%%%%%%%%%%%%%%%%%%%%%%%%

%%%%%%%%%%%%%%%%%%%%%%%%%%%%%%%%%%%%%%%%%%%%%%%%%%%%%%%%%%%%
%\chapter{Dialogbeispiele}\label{chapter:dialogs}
%%%%%%%%%%%%%%%%%%%%%%%%%%%%%%%%%%%%%%%%%%%%%%%%%%%%%%%%%%%%
\section{Beschreibung einer Simulation}\label{sec:sim_description}
Zusammenfassendes Beispiel der Erstellung einer Simulation mit Hilfe von handgeschriebenen Java-Generatoren.\\
Dazu: ein UML Diagramm das die Abhängigkeiten der schon oben gezeigten Module untereinander zeigt und einleitet warum wir uns für diese feingliedrige Erstellung entschieden haben.\\
\section{Interaktion mit der SimpleGui}\label{sec:dialog_simplegui}


\imiscomment{Darstellung des Systems anhand von Beispieldialogen mit Abbildungen und Erläuterungen}

\imiscomment{Ausgiebige Verwendung von Fotos und Bildschirm-Harcopies}

\imiscomment{Bei Farbbildern sollte sichergestellt werden, dass diese auch in Schwarz-Weiss gut erkennbar sind, da selbst bei Produktion einer Arbeit in Farbe später eventuell Kopien angefertigt werden}

%%%%%%%%%%%%%%%%%%%%%%%%%%%%%%%%%%%%%%%%%%%%%%%%%%%%%%%%%%%%
%\section{Dialogbeispiel 1}\label{sec:dialog_1}
%%%%%%%%%%%%%%%%%%%%%%%%%%%%%%%%%%%%%%%%%%%%%%%%%%%%%%%%%%%%
%
% pdfLaTeX can use png, jpeg, etc:
% \begin{figure}[htb]
%   \begin{center}
%     \includegraphics[width=\textwidth]{pics/smartboardTablet}
%   \end{center}
%   \caption{Smartboard Software auf dem Tablett}
%   \label{fig:smartboardTablet}
% \end{figure}    

%%%%%%%%%%%%%%%%%%%%%%%%%%%%%%%%%%%%%%%%%%%%%%%%%%%%%%%%%%%%
%\section{Dialogbeispiel 2}\label{sec:dialog_2}
%%%%%%%%%%%%%%%%%%%%%%%%%%%%%%%%%%%%%%%%%%%%%%%%%%%%%%%%%%%%

%%%%%%%%%%%%%%%%%%%%%%%%%%%%%%%%%%%%%%%%%%%%%%%%%%%%%%%%%%%%
%%%%%%%%%%%%%%%%%%%%%%%%%%%%%%%%%%%%%%%%%%%%%%%%%%%%%%%%%%%%

\chapter{Evaluation}\label{chapter:eval}
%%%%%%%%%%%%%%%%%%%%%%%%%%%%%%%%%%%%%%%%%%%%%%%%%%%%%%%%%%%%

%%%%%%%%%%%%%%%%%%%%%%%%%%%%%%%%%%%%%%%%%%%%%%%%%%%%%%%%%%%%
\section{Ziel}\label{sec:eva_goal}
%%%%%%%%%%%%%%%%%%%%%%%%%%%%%%%%%%%%%%%%%%%%%%%%%%%%%%%%%%%%

\imiscomment{Was soll evaluiert werden und warum\ldots}

%%%%%%%%%%%%%%%%%%%%%%%%%%%%%%%%%%%%%%%%%%%%%%%%%%%%%%%%%%%%
\section{Vorgehen}\label{sec:eva_approach}
%%%%%%%%%%%%%%%%%%%%%%%%%%%%%%%%%%%%%%%%%%%%%%%%%%%%%%%%%%%%

\imiscomment{Welches Vorgehen wurde bei der Evaluierung gewählt und warum\ldots}

\imiscomment{Hier bietet sich der Einsatz von Diagrammen und Schaubildern an.}

%%%%%%%%%%%%%%%%%%%%%%%%%%%%%%%%%%%%%%%%%%%%%%%%%%%%%%%%%%%%
\section{Methoden}\label{sec:eva_methods}
%%%%%%%%%%%%%%%%%%%%%%%%%%%%%%%%%%%%%%%%%%%%%%%%%%%%%%%%%%%%

\imiscomment{Welches Methoden kamen zum Einsatz und warum\ldots}

%%%%%%%%%%%%%%%%%%%%%%%%%%%%%%%%%%%%%%%%%%%%%%%%%%%%%%%%%%%%
\section{Ergebnisse}\label{sec:eva_results}
%%%%%%%%%%%%%%%%%%%%%%%%%%%%%%%%%%%%%%%%%%%%%%%%%%%%%%%%%%%%

\imiscomment{Welche Ergebnisse brachte die Evaluierung und was ist davon zu halten\ldots}

\imiscomment{Hier helfen Tabellen und Grafiken beim Vermitteln der Sachverhalte.}


%%%%%%%%%%%%%%%%%%%%%%%%%%%%%%%%%%%%%%%%%%%%%%%%%%%%%%%%%%%%
%%%%%%%%%%%%%%%%%%%%%%%%%%%%%%%%%%%%%%%%%%%%%%%%%%%%%%%%%%%%

%!TEX root = /Users/marv/devel/multimediaprogrammierung/paper/documentation.tex
\chapter{Zusammenfassung und Ausblick}\label{chapter:conclusions}
%%%%%%%%%%%%%%%%%%%%%%%%%%%%%%%%%%%%%%%%%%%%%%%%%%%%%%%%%%%%
In diesem Kapitel wird kurz aufgezeigt, was in dieser Arbeit erreich wurde, welche Punkt dabei offen geblieben sind und in welcher Weise man diese in Zukunft angehen könnte.
%%%%%%%%%%%%%%%%%%%%%%%%%%%%%%%%%%%%%%%%%%%%%%%%%%%%%%%%%%%%
\section{Zusammenfassung}\label{sec:conc_summary}
%%%%%%%%%%%%%%%%%%%%%%%%%%%%%%%%%%%%%%%%%%%%%%%%%%%%%%%%%%%%
Wir haben es geschafft einen Simulator zu entwickeln, der das MACK-Framework testet. Die Methode des benutzerzentrierten Designs war sehr erfolgreich. Durch die vielen, frühzeitigen Rückmeldung der späteren Benutzer konnten wir uns immer schnell an Wünsche und Anregungen anpassen. So haben wir vermieden, dass sich nach der Arbeitsphase am Projekt noch sehr aufwendige Änderungen ergeben. 
In diversen Treffen untereinander und mit unserem wissenschaftlichen Begleiten haben wir eine strukturierte Arbeitsweise entwickelt. Diese half uns, das Projektziel in einer Art und Weise zu erreichen die die Entwicklung für uns, als auch für zukünftige Benutzer und Entwickler der Software, besonders angenehm machte und machen wird.\\

Unsere Evaluation hat ergeben, dass unser System im Wesentlichen funktionsfähig ist und einen guten Ansatzpunkt für weitere Arbeiten an diesem Projekt bietet. Der Ablauf der Simulation wirkt auf uns ausreichend realistisch und sollte sich gut zum Testen des MACK-Frameworks eignen. Die Sensoren und Aktuatoren ändern ihre Werte in Abhängigkeit vom Geschehen und nehmen ebenfalls Einfluss darauf.

\imiscomment{Darstellung, was erreicht wurde (ca. 1 Seite) Aber ich weiß einfach nicht so recht wie und was wir hier noch ausschmücken sollten!}

%%%%%%%%%%%%%%%%%%%%%%%%%%%%%%%%%%%%%%%%%%%%%%%%%%%%%%%%%%%%
\section{Offene Punkte}\label{sec:conc_open_questions}
%%%%%%%%%%%%%%%%%%%%%%%%%%%%%%%%%%%%%%%%%%%%%%%%%%%%%%%%%%%%
Im Rahmen unseres Projektes konnten nicht alle der ursprünglich geplanten Punkte umgesetzt werden. Einzig die Beschreibungssprache CASiX wurde komplett gestrichen.

\subsection{CASiX}\label{subsec:conc_open_questions_casix}
Eine in der Analyse festgestellte Anforderung an unsere Software war die Möglichkeit, Simulationen durch eine Beschreibungssprache zu erzeugen. Im frühen Stadium unserer Arbeit haben wir uns, mit Hilfe der befragten Benutzer, für XML entschieden. Es sollte eine XML-Spezifkiation in Form einer XML-Schema-Datei festgelegt werden. Unter dem Namen \codeinline{CASiX} wurde ein erster Version spezifiziert.\\

Im weiteren Entwicklungsprozess wurde jedoch festgestellt, dass die Spezifikation in ihrer vorliegenden Form nicht ansatzweise flexibel genug war, um unseren Anspruch an die gebrauchstaugliche Gestaltung einer Simulation zu genügen. Daraufhin wurde der Schritt wieder zurück zur Spezifikation gegangen. Nach langer Diskussion haben wir uns schlussendlich dafür entschieden, unsere Arbeit ohne die Beschreibungssprache fortzusetzen. Da wir nicht nur in deren Planung, sondern, vor allem, in deren späteren Implementierung im System große Schwierigkeiten sahen.
Ungefähr zur Projektmitte haben wir die Simulations\-beschreibungs\-sprache in den letzen Projekt-Meilenstein verlegt. Dieser enthält Ideen und Konzepte die wir uns für die Zukunft von CASi vorstellen können, im Rahmen dieses Projekts aber keine Zeit dafür gefunden werde konnte. Wir sind der Meinung, dass die durch diese Entscheidung eingesparte Arbeit in unserem Projekt an anderen Stellen sinnvoller zum Einsatz kam.\\

Durch den stark modularen Aufbau der Software ist es jedoch für spätere Entwickler besonders einfach diesen Aspekt noch einmal aufzugreifen und in CASi zu integrieren.

\subsection{Antwortzeiten des MACK-Servers}\label{subsec:conc_open_questions_mack}
Ein aktuelles Problem ist, dass bei schneller Simulationsgeschwindigkeit die Antworten des MACK-Servers zu stark verzögert sind. In diesem Fall passen die Sensor- und Aktuatorwerte nicht zum aktuellen Verhalten der Simulation. Diese können sich auf den gesamten Simulationslauf auswirken. Dagegen können wir uns zwei unterschiedlich einflussreiche Verbesserungen vorstellen. 

Die Kommunikationskanäle hinsichtlich ihrer Geschwindigkeit optimieren. Den Jabber-Host (über den sämtliche Kommunikation läuft) lokal auf Maschine des Awareness-Hubs zu installieren verringert die Paketlaufzeiten vermutlich erheblich.

Weit sinnvoller halten wir es jedoch, das MACK-Protokoll zu überarbeiten. Wenn dieses dahingehend erweitert wird, das Zeitstempel und oder andere Identifikationsmerkmale übermittelt werden, könnten sich der Awareness-Hub und der Simulator unabhängig ihrer Geschwindigkeit austauschen. Reasoner für die zeitliche Aspekte wichtig sind, wären in der Lage auf die erhöht Geschwindigkeit der Simulation zu versehen und sich darauf gehend anzupassen.

%%%%%%%%%%%%%%%%%%%%%%%%%%%%%%%%%%%%%%%%%%%%%%%%%%%%%%%%%%%%
\section{Ausblick}\label{sec:conc_outlook}
%%%%%%%%%%%%%%%%%%%%%%%%%%%%%%%%%%%%%%%%%%%%%%%%%%%%%%%%%%%%
Aspekte unserer Arbeit an die gut in der Zukunft angeknüpft werden kann sind prinzipiell in zwei Teilbereiche aufzuteilen. Eine Weiterentwicklungsmöglichkeit ist es eigene, neue Simulationen zu entwerfen. Die dazu benötigten anderen Sonsoren, Aktuatoren oder Agentverhalten wären unkompliziert zu erstellen. Man könnte sich an das in dieser Dokumentation beschriebene Vorgehen halten, oder sich an unserer Demonstations\-simulation orientieren.\\

Zum anderen sind auch für den CASi-Kern Erweiterungen vorstellbar. Dazu zählen insbesondere die Entwicklung einer umfangreicheren graphischen Schnittstelle, zum Beispiel durch eine drei dimensionale Darstellung der Simulations-Welt. Weitergehen könnten wir uns vorstellen, Simulationen interaktiv zu gestalten. Denkbar wäre  eine Simulation in einer virtuellen Realität, in der die Benutzer die Agenten direkt steuern können. Dafür könnten Netzwerk-Mehrspieler-Spiele genutzt werden. Beispielsweise Minecraft \cite{web-minecraft} bietet dafür eine in Java programmierte quelloffene Serverkomponente an, welcher sich durch geeignete Schnittstellen erweitern lassen könnte.

%%%%%%%%%%%%%%%%%%%%%%%%%%%%%%%%%%%%%%%%%%%%%%%%%%%%%%%%%%%%

%%%%%%%%%%%%%%%%%%%%%%%%%%%%%%%%%%%%%%%%%%%%%%%%%%%%%%%%%%%%
% list of figures, tables
%%%%%%%%%%%%%%%%%%%%%%%%%%%%%%%%%%%%%%%%%%%%%%%%%%%%%%%%%%%%

% now we have all the stuff where chapters have no numbers etc.
% you will find a lot of \cleardoublepage and \phantomsection
% commands, these help hyperref.sty to find the right targets
% for hyperlinks
\backmatter

% The list of figures
  \cleardoublepage
  \phantomsection
  \addcontentsline{toc}{chapter}{Abbildungen}
  \listoffigures

% The list of tables
  \cleardoublepage
  \phantomsection
  \addcontentsline{toc}{chapter}{Tabellen}
  \listoftables

% list of listings
  \cleardoublepage
  \phantomsection
  \addcontentsline{toc}{chapter}{Quelltexte}
%   \lstlistoflistings  % generated by listings.sty
  \listofimislistings % generated by tocloft.sty

%%%%%%%%%%%%%%%%%%%%%%%%%%%%%%%%%%%%%%%%%%%%%%%%%%%%%%%%%%%%
\chapter{Quellen}
% \addcontentsline{toc}{chapter}{Quellen}
%%%%%%%%%%%%%%%%%%%%%%%%%%%%%%%%%%%%%%%%%%%%%%%%%%%%%%%%%%%%
\phantomsection
\renewcommand{\bibname}{Literatur}
\bibliographystyle{imis}
\bibliography{bibliography}
\addcontentsline{toc}{section}{Literatur}

\phantomsection
\bibliographystyleweb{imis}
\bibliographyweb{bibliography}
\addcontentsline{toc}{section}{Weblinks}

%
%\section*{Software}
%\addcontentsline{toc}{section}{Software}
%
%openSUSE 11.3, 15. Juli 2010, \url{http://www.opensuse.org/}

%%%%%%%%%%%%%%%%%%%%%%%%%%%%%%%%%%%%%%%%%%%%%%%%%%%%%%%%%%%%
% list of abbreviations. glossaries
%%%%%%%%%%%%%%%%%%%%%%%%%%%%%%%%%%%%%%%%%%%%%%%%%%%%%%%%%%%%

% creating the list of abbreviations and the glossary
% we just add all entries in the abk and glos files
% this is really just the easiest way of doing it, the glossaries.sty
% package has much more option to fine-tune the results

\glsaddall
% \printglossary[type=\acronymtype,title=Abkürzungen,style=imisabk]
\printglossary[type=\acronymtype,title=Abkürzungen,style=imisabk,toctitle=Abkürzungen]
\printglossary[style=imis]

% % If we want to generate an index, this should be it
%   \cleardoublepage
%   \phantomsection
% %   \addcontentsline{toc}{chapter}{Citation Index} % only used for non-KOMA classes
%   \sloppy
%   \printindex
%   \fussy


%%%%%%%%%%%%%%%%%%%%%%%%%%%%%%%%%%%%%%%%%%%%%%%%%%%%%%%%%%%%
% we now go into the appendix. basically the appendix is one chapter
% with some sections below. The appendix chapter has now mark, the
% sections alphabetical one, subsections will have numerical
% TODO format lower level sections in appendix
\renewcommand{\thechapter}{\Alph{chapter}}
\renewcommand{\thesection}{\Alph{section}}
\renewcommand{\thesubsection}{\Alph{section}.\arabic{subsection}}
\setcounter{section}{0} % need to be explicit, since they are not reset in backmatter

%%%%%%%%%%%%%%%%%%%%%%%%%%%%%%%%%%%%%%%%%%%%%%%%%%%%%%%%%%%%
\chapter{Anhänge}\label{chapter:appendix}
% \addcontentsline{toc}{chapter}{Anhänge}
%%%%%%%%%%%%%%%%%%%%%%%%%%%%%%%%%%%%%%%%%%%%%%%%%%%%%%%%%%%%

\imiscomment{Umfangreiche zusätzliche Informationen, die im Textverlauf stören würden, aber für die Arbeit wichtig sind, wie z.B. Programmcode, Fragebögen, Evaluationstabellen}

%%%%%%%%%%%%%%%%%%%%%%%%%%%%%%%%%%%%%%%%%%%%%%%%%%%%%%%%%%%%
\section{Programmcode}\label{sec:appendix_A}
%%%%%%%%%%%%%%%%%%%%%%%%%%%%%%%%%%%%%%%%%%%%%%%%%%%%%%%%%%%%

%%%%%%%%%%%%%%%%%%%%%%%%%%%%%%%%%%%%%%%%%%%%%%%%%%%%%%%%%%%%
\subsection{Modul 1}
%%%%%%%%%%%%%%%%%%%%%%%%%%%%%%%%%%%%%%%%%%%%%%%%%%%%%%%%%%%%

%%%%%%%%%%%%%%%%%%%%%%%%%%%%%%%%%%%%%%%%%%%%%%%%%%%%%%%%%%%%
\subsection{Modul 2}
%%%%%%%%%%%%%%%%%%%%%%%%%%%%%%%%%%%%%%%%%%%%%%%%%%%%%%%%%%%%

%%%%%%%%%%%%%%%%%%%%%%%%%%%%%%%%%%%%%%%%%%%%%%%%%%%%%%%%%%%%
\section{Projektstruktur}\label{sec:appendix_structure}
Um die Weiterentwicklung des Projektes zu erleichtern, befindet sich in diesem Abschnitt eine ausführliche Erklärung zur Struktur des Projekts. Dies beinhaltet zum einen die Paket- und Ordnerstruktur, aber auch Hinweise auf Funktionen, die die Entwicklung erleichtern.
\subsection{Ordnerstruktur}
Das Projekt ist in verschiedene Ordner unterteilt, deren Sinn nachfolgend erläutert wird:
\begin{description}
	\item[src] In diesem Ordner befinden sich die Beschreibungen und Implementierungen des Simulators und der Simulationen.
	\item[test] Dieser Ordner enthält die JUnit-Testfälle für einige Klassen im \texttt{src}-Ordner. Hierbei entspricht die interne Paket-Struktur dieses Ordners der Struktur des \texttt{src}-Ordners.
	\item[log] Logfiles, die während der Ausführung des Programms generiert werden, werden mit einem Zeitstempel in diesem Ordner abgelegt.
	\item[doc] Die JavaDoc-Dokumentation des Simulators befindet sich in diesem Verzeichnis.
	\item[dist] Im Distributionsverzeichnis finden sich ausführbare Versionen der Anwendungen als jar-Datei.
	\item[sims] Beschreibungen für Simulationen, wie z.B. textuelle Erklärungen zu einzelnen Simulationen und die Netzwerk-Konfigurationsdateien befinden sich in diesem Ordner.
	\item[bin] Binaries des Simulators
	\item[lib] externe Bibliotheken
	\item[reports] Reports über die Durchführung der Testfälle
\end{description}
\subsection{Paketstruktur}
Alle Klassen, die zum Simulator gehören befinden sich in Unterpaketen des Paketes \texttt{de.uniluebeck\-.imis.casi} im Ordner \texttt{src}.
Nachfolgend wird der Sinn der verschiedenen Pakte erläutert:
\begin{description}
	\item[engine] Dieses Paket enthält die Basis des Simulators, die für die Koordination der Ausführung und die Verknüpfung der Komponenten sorgt. Hierzu zählen insbesondere die \texttt{SimulationEngine} und die \texttt{SimulationClock}.
	\item[communication] In diesem Paket befinden sich die Schnittstellenbeschreibungen für Kommunikationshandler. In Unterpaketen befinden sich konkrete Implementierungen. So enthält \texttt{com\-munication\-.mack} die Implementierung des \texttt{MACKNetworkHandler}s.
	\item[generator] Das \texttt{Generator}-Paket beinhaltet die Beschreibung der Schnittstellen, die von Generatoren der Welt implementiert werden müssen.
	\item[ui] In diesem Paket befinden sich die Beschreibungen für Benutzungsschnittstellen. Unterpakete enthalten die Implementierungen verschiedener Userinterfaces. Das Paket \texttt{ui.simplegui} enthält die einfache graphische Benutzungsoberfläche.
	\item[utils] Im \texttt{utils}-Paket befinden sich Klassen, die einen hohen Abstraktionsgrad aufweisen und somit leicht in andere Projekte portiert werden können, da sie größtenteils vom Simulator unabhängig sind.
	\item[simulations] Dieses Paket kapselt in weiteren Unterpaketen die Beschreibungen einzelner Simulationen.
	\item[simulation.model] Die Anwendungslogik befindet sich in diesem Paket.
	\item[controller] Die Kontroller zur Verknüpfung der Komponenten und zur Steuerung des Programmflusses befinden sich in diesem Paket.
	\item[logging] Dieses Paket beinhaltet Konfigurationsklassen für die Logger.
\end{description}
\subsection{Buildfile}
Das Projektverzeichnis enthält das Ant-Buildfile \texttt{build.xml}, welches eine Reihe von Buildtargets definiert. Diese Ziele können entweder über die Konsole mit \texttt{ant <target-name>} oder aus Eclipse heraus ausgeführt werden. Nachfolgend befindet sich eine Beschreibung der wichtigsten Targets:
\begin{description}
	\item[clean] räumt das Projektverzeichnis auf und löscht die Binarie-, Dokuentations- und Reportverzeichnisse.
	\item[clean-all] führt das Target \texttt{clean} aus und löscht außerdem das Log- und Distributionsverzeichnis.
	\item[test] führt die Testfälle aus.
	\item[test] führt die Testfälle aus und erzeugt HTML-Reports im \texttt{reports}-Verzeichnis.
	\item[doc] erzeugt die JavaDoc-Dokumentation
	\item[jar] erzeugt ein ausführbares jar-File des Simulators (\texttt{dist/CASi.jar})
	\item[xmpp-registrator-jar] erstellt eine ausführbare Version des \texttt{XmppRegistrator}s.
	\item[main] Erstellt alle Distributionen und die JavaDoc-Dokumentation.
\end{description}
%%%%%%%%%%%%%%%%%%%%%%%%%%%%%%%%%%%%%%%%%%%%%%%%%%%%%%%%%%%%

%%%%%%%%%%%%%%%%%%%%%%%%%%%%%%%%%%%%%%%%%%%%%%%%%%%%%%%%%%%%
\section{Ausführungsanweisungen}\label{sec:appendix_install}
Da es sich bei dem Projekt um ein Java-Projekt handelt, gibt es keine Installation im herkömmlichen Sinne. In diesem Abschnitt beschrieben, wie die Programme ausgeführt werden können.
\subsection{Simulator}
Die jar-Version des Simulators kann, nachdem sie mit \texttt{ant jar} erzeugt wurde, mit \texttt{java -jar dist/CASi.jar <parameter>} ausgeführt werden. Eine ausführliche Dokumentation der möglichen Parameter erhält man mit \texttt{java -jar dist/CASi.jar \texttt{-}\texttt{-}help}.
\subsection{XmppRegistrator}
Der XmppRegistrator kann automatisch alle für die Simulation benötigten XmppIdentifier am Jabber-Server registrieren. Dies ist besonders hilfreich, wenn der Server ein Delay zwischen zwei Registrierungsoperationen benötigt, da somit vermieden werden kann, dass die Identifier erst beim Start der Simulation registriert werden und somit die Ausführung der Simulation verzögert wird.

Der XmppRegistrator verlangt als einzigen Parameter den Pfad zu einer Netzwerkkonfigurationsdatei und kann, nachdem er mit \texttt{ant xmpp-registrator-jar} erzeugt wurde, mit \texttt{java -jar dist/XmppRegistrator.jar <path-to-config-file>} ausgeführt werden.
%%%%%%%%%%%%%%%%%%%%%%%%%%%%%%%%%%%%%%%%%%%%%%%%%%%%%%%%%%%%

%%%%%%%%%%%%%%%%%%%%%%%%%%%%%%%%%%%%%%%%%%%%%%%%%%%%%%%%%%%%
\section{Evaluationsergebnisse}\label{sec:appendix_C}
%%%%%%%%%%%%%%%%%%%%%%%%%%%%%%%%%%%%%%%%%%%%%%%%%%%%%%%%%%%%

%%%%%%%%%%%%%%%%%%%%%%%%%%%%%%%%%%%%%%%%%%%%%%%%%%%%%%%%%%%%
\chapter*{Erklärung}\addcontentsline{toc}{chapter}{Erklärung}
\thispagestyle{empty}
%%%%%%%%%%%%%%%%%%%%%%%%%%%%%%%%%%%%%%%%%%%%%%%%%%%%%%%%%%%%
% you did it:) now the only thing left is to sign and deliver it
Ich versichere, die vorliegende Arbeit selbstständig verfasst und nur die
angegebenen Quellen benutzt zu haben.


\vspace*{5cm}
Lübeck, den \rule{0.3\textwidth}{0.4pt} \hspace*{1cm} \rule{0.3\textwidth}{0.4pt}
\end{document}
